\chapter{Propuesta}\label{chapter:proposal}
	La propuesta de solución consiste en varios pasos. Primero se capturan las señales de los sensores y se almacenan.
	Luego, se realiza un preprocesado de los datos, para eliminar datos que sean erróneos. Durante este proceso, también
	se extraen ciertas características importantes a partir de las ya existentes en los datos obtenidos durante la fase
	de captura de señales. Una vez se tengan los datos que se quieren estructurados, se pasa a una fase de detección de
	anomalías utilizando algunos métodos heurísticos y de aprendizaje no-supervisado. Las anomalías que se extraigan con
	este proceso son etiquetadas con un método que utiliza la distancia de la ubicación de las lecturas a ciertas marcas
	previamente etiquetadas a mano. Luego se realiza un proceso de selección de características utilizando varios métodos, 
	para determinar las características más relevantes a la hora de clasificar las anomalías en bache o no bache. Y
	finalmente se realiza un proceso de selección de modelos e hiperparámetros, mediante el cual se busca un algoritmo
	y una configuración de hiperparámetros que maximice el \emph{F1 score}.

\section{Captura de señales}
	Las señales que se consideraron capturar son las proporcionadas por sensores como el \textbf{acelerómetro},
	el \textbf{giroscopio} y el \textbf{GPS}, pues son los sensores más comunes en los teléfonos inteligentes. El
	\textbf{acelerómetro}, como se explica en la literatura, permite medir en tiempo real la aceleración(rapidez de
	cambio de la \textbf{velocidad}) en $m/s^2$ con respecto a los 3 ejes principales, y en el caso de los
	acelerómetros integrados en los teléfonos inteligentes permite medir la aceleración en los 3 ejes
	principales del dispositivo en cuestión. Estos 3 ejes se encuentran ubicados, si se sostiene el
	dispositivo de la forma usual, de la siguiente manera:

	\begin{itemize}
		\item eje X: Va de un lado del dispositivo a otro.
		\item eje Y: Va desde la parte inferior hasta la parte superior del dispositivo.
		\item eje Z: Es ortogonal a la pantalla del dispositivo.
	\end{itemize}
	
	\begin{figure}[htb]
		\centering
		\includegraphics[scale = 0.5]{Graphics/mobile_phone_axis.png}
		\caption{Ejes de un dispositivo móvil}
		\label{fig:4}
	\end{figure}

	\subsection{Acelerómetro}
		La señal del \textbf{acelerómetro} mientras el vehículo se desplaza por carreteras en buen estado es bien similar,
		sin embargo una vez pasa por encima de un bache ocurre un evento anómalo en el que el vehículo se ve afectado
		por el bache. En ese momento cae brevemente e incluso se inclina momentáneamente hacia un lado (en el caso de un bache
		que afecte una sola rueda). Por esta razón la señal del \textbf{acelerómetro} en un intervalo de tiempo determinado, debe tener
		un comportamiento fuera de lo común que permita identificar el evento en la serie temporal capturada por este sensor.
		Tal como se plantea en la literatura, la característica fundamental de un bache son lecturas anormales en el eje
		Z del \textbf{acelerómetro} durante un intervalo de tiempo determinado. También, dicho sensor debe registrar lecturas
		anormales en el eje X si el bache afecta un solo lado del vehículo, pues en este caso el vehículo se inclina violentamente
		hacia un lado por un momento.\\

	\begin{figure}[htb]
		\centering
		\includegraphics[scale = 0.5]{Graphics/one_side_pothole_vehicle.jpg}
		\caption{Vehículo impactado por un bache en un solo lado}
		\label{fig:5}
	\end{figure}

	\subsection{Giroscopio}
		El \textbf{giroscopio} es otro de los sensores que se considera que aportaría información útil, ya que este sensor 
		mide en $rad/s$ la velocidad angular del dispositivo móvil con respecto a los 3 ejes mencionados previamente. 
		En el momento que un vehículo interactúa con un bache ocurren vibraciones que hacen que el vehículo gire, lo que
		debería generar picos en las lecturas del \textbf{giroscopio} en un intervalo de tiempo. El problema con este sensor es 
		que, a pesar de ser de los más comunes, no es abundante en teléfonos inteligentes de gama baja por lo que 
		su uso en la propuesta de solución se tiene pensado como algo que permita mejorarla y no algo de lo que esta
		dependa.\\

	\subsection{GPS}
		El \textbf{GPS}(\emph{Global Positioning System}) es el sensor que le permite al dispositivo móvil ubicarse sobre la
		superficie de la tierra utilizando coordenadas de latitud-longitud. Debido a que este sistema utiliza señales para
		comunicarse con satélites en tiempo real, existe una latencia debido al tiempo que le toma a la señal viajar hacia el
		satélite y de vuelta al dispositivo. Además, las condiciones atmosféricas pueden afectar esta latencia, así como el hecho
		de que la señal puede ser reflejada por el terreno, edificios, etc., como por ejemplo cuando se viaja a través de un
		túnel subterráneo donde la señal no puede llegar. También es importante destacar que las lecturas que genera el \textbf
		{GPS} tienen cierto margen de error que varía de un dispositivo a otro y que depende de la potencia del sensor en sí,
		pero en general según la literatura la mediana del error en estos sensores es de 5m aproximadamente \parencite{eriksson2008pothole},
		ambos aspectos son importantes a tener en cuenta al hacer uso de este sensor. El mismo se puede utilizar para estimar la
		\textbf{velocidad} a la que viaja el vehículo, lo que es necesario ya que la \textbf{velocidad} es un factor que influye de
		forma directa en la forma que un vehículo es afectado por un bache, además con este sensor es posible ubicar en un mapa
		aproximadamente los lugares donde existen baches u otro tipo de anomalías, incluso puede ser útil para marcar rutas de viaje
		y así facilitar algunos aspectos de la propuesta de solución.

	\subsection{Velocidad y frecuencia de muestreo}
		Debido a que la \textbf{velocidad} de un vehículo durante un recorrido no suele ser constante, es necesario que la \textbf
		{frecuencia de muestreo} cambie en función de la \textbf{velocidad} a la que vaya el vehículo. Pues si el vehículo se
		desplaza más rápido será necesario muestrear más para mantener la precisión y consistencia de los datos que se generen.\\
		La frecuencia de muestreo se tomó de tal forma que se tuviera una lectura de \textbf{acelerómetro} y \textbf
		{giroscopio} por cada metro. Si el vehículo se desplaza a una \textbf{velocidad} $v$ en $m/s$ y se quiere obtener
		$x$ muestras por cada metro, quiere decir que se necesitan obtener $v * x$ muestras, por lo que la frecuencia de muestreo
		necesaria sería de $v * x Hz$. Como se quiere una sola muestra por cada metro la frecuencia de muestreo
		en este caso particular es de $v Hz$. La actualización de la \textbf{frecuencia de muestreo} ocurre cada vez que se recomputa
		la velocidad, lo que se propuso realizar cada 5 segundos debido a la posible latencia inherente del \textbf{GPS} a la hora
		obtener las lecturas.\\
		Para actualizar de forma correcta la \textbf{frecuencia de muestreo} es necesario conocer la \textbf{velocidad}
		a la que viaja el vehículo, o al menos realizar una estimación aceptable de la misma utilizando los sensores de
		los que dispone el dispositivo móvil. Esta estimación se realizó utilizando las lecturas del \textbf{GPS}, y se 
		asignaron frecuencias de muestreo teniendo en cuenta los límites de velocidad existentes en la zona urbana donde 
		se llevaron a cabo los experimentos. Debido a que en Cuba existen muy pocas carreteras donde se permita a
		los vehículos desplazarse a más de 100 km/h y también debido a que la mayoría de los vehículos que circulan en
		el país no alcanzan dicha \textbf{velocidad}, se asumió que esta es la máxima a la que podría desplazarse un
		vehículo a la hora de decidir las frecuencias de muestreo de los sensores.\\

	\subsection{Posición del móvil en el vehículo}
		La posición en la que se coloca el dispositivo móvil en el vehículo es relevante, al menos por ahora, pues la
		dirección del eje Y del vehículo debe coincidir con la del eje Y del móvil para que las señales que se capturen
		se correspondan con la realidad. De lo contrario, se deberían realizar ciertos ajustes utilizando algún método de
		transformación de coordenadas para poder obtener lecturas acertadas sin importar la posición en la que se coloque
		el dispositivo.

\section{Detección de anomalías}
	Los baches no son más que anomalías en la carretera. Teniendo en cuenta esto, se propuso detectar dichar anomalías utilizando
	2 enfoques. Primero, intentar identificar las anomalías utilizando los métodos heurísticos sugeridos en la literatura basados
	en umbrales. Y segundo emplear algoritmos de aprendizaje de máquinas para detectar anomalías.\\
	\indent Los métodos heurísticos que se utilizaron en la propuesta fueron \textbf{Z-THRESH}, \textbf{Z-DIFF} y \textbf{G-ZERO}.
	Estos solo hacen uso de las lecturas del \textbf{acelerómetro} para determinar la existencia de una anomalía. Por otro lado los
	métodos de aprendizaje de máquinas que se probaron fueron \textbf{DBSCAN}, \textbf{OPTICS} y \textbf{One Class SVM} utilizando 
	6 características, las lecturas de los 3 ejes de \textbf{acelerómetro}, y además las lecturas de los 3 ejes del \textbf{giroscopio}.
	Este último sensor se incluyó, pues se consideró que sus lecturas pueden ser significativas a la hora de determinar la existencia de una 
	anomalía. Este proceso de detección de anomalías se aplicó a cada una de las series temporales obtenidas como resultado del muestreo
	de los sensores del dispositivo móvil.

	%Hablar sobre los hiperparámetros seleccionadas para cada uno de los métodos.

\section{Clasificación de outliers}

	%------------------------- Detallar que framework utilizamos para capturar los sensores ---------------------------
	% Primero que todo era necesario tener alguna forma de obtener señales de los sensores necesarios
	% para poder llevar a cabo nuestra propuesta. Para esto se apredió a trabajar en \textbf{Flutter} y
	% se hizo una aplicación móvil sencilla utilizando algunas bibliotecas propias de ese
	% \emph{framework} que nos permitiera obtener las señales de los sensores y exportarlas en un
	% {\textbf .json} para luego poder trabajar con dichos datos en el modelo de \emph{Machine Learning}.

