\chapter*{Introducción}\label{chapter:introduction}
\addcontentsline{toc}{chapter}{Introducción}

Cuba se ha propuesto, desde hace unos años, llevar a cabo un profundo proceso de informatización.
Transacciones bancarias, datos móviles, digitalización de información, acceso a información, etc. Todos
estos han sido, y aún son, temas dentro de dicho proceso con el objetivo de facilitar muchas actividades
cotidianas. En los tiempos actuales las carreteras son algo común y los vehículos que hacen uso de estas
también. No obstante, a menudo las carreteras se deterioran debido al clima, al tráfico o a la mala calidad 
de los distintos tipos de trabajo que se llevan a cabo en la vía pública, en Cuba es algo bastante común que
la mayoría de las carreteras en las grandes ciudades, como La Habana, estén en malas condiciones.\\
\indent En la isla no se conoce ninguna forma de ubicar de forma sencilla y accesible los tramos de carreteras que están en
mal estado. Sería muy útil poder hacer esto pues de esta forma se podría tener acceso a esta información de forma
instantánea. El acceso rápido a estos datos no solo facilitaría la reparación de estas calles, sino que también
podría servir de guía a choféres para evitar tales tramos y por tanto consecuencias ocasionadas por estas condiciones
como accidentes y daños a la estructura de los vehículos. Además, el hecho de no tener que enviar vehículos a
inspeccionar el estado de las carreteras ya constituye un ahorro importante de recursos y de tiempo, y permitiría
tomar decisiones con mayor rapidez sobre que tramos de carretera priorizar.\\
\indent El problema que por tanto se pretende resolver con este trabajo es determinar cuando hay un bache en la carretera, y en que
posición se encuentra valiéndose de un dispositivo móvil.\\
\indent Para resolver este problema se explorará la factibilidad de un modelo de \emph{Machine Learning} para detectar y clasificar
las anomalías en la carretera.\\
\indent La propuesta en cuestión se basa en los datos que se obtienen con ciertos sensores y que por lo general no vienen integrados
en la mayoría de los vehículos, además, sería más incómodo adaptar estos sensores por separado a un vehículo que por ejemplo, utilizar
un dispositivo móvil que ya viene con los sensores integrados en su \emph{hardware}. Cada vez se hace más común que gran parte de 
las personas tengan un dispositivo móvil y lo usen para realizar distintas actividades. Esto permite que sus dispositivos puedan ser utilizados
para obtener los datos necesarios a partir de las lecturas que devuelven los sensores que tienen integrados, haciendo factible la propuesta. 
Cualquier persona con su dispositivo móvil podría contribuir a la propuesta de solución y beneficiarse de la misma.\\
\indent Muchos de los primeros trabajos que se propusieron resolver este problema utilizaban métodos de detección que se basaban en fijar
umbrales, y que a pesar de funcionar no tenían muy buena precisión[~\brackcite{eriksson2008pothole}, ~\brackcite{mohan2008nericell},
~\brackcite{mednis2011real}]. Más recientemente con el auge del aprendizaje de máquinas muchos autores han atacado este problema 
utilizando las bondades de dicha rama, extrayendo varios \emph{features} de dominio temporal y de frecuencia, y utilizando métodos
como \emph{Support Vector Machines}(\textbf{SVM}), redes neuronales artificiales(\textbf{ANN}) y árboles de decisión (\textbf{DT})
[~\brackcite{el2018towards}, ~\brackcite{seraj2015roads}, ~\brackcite{gonzalez2017learning}, ~\brackcite{zheng2020fused},
~\brackcite{perttunen2011distributed}]. Muchos autores han tenido en cuenta además la naturaleza contaminada por ruido de los datos
obtenidos utilizando estos dispositivos y han utilizando técnicas de procesamiento de señales digitales para mejorar la calidad de
la señal obtenida con el objetivo de obtener datos más significativos que caractericen de forma más precisa la señal y por lo tanto mejoren la
calidad de las predicciones de los modelos de \emph{Machine Learning}[~\brackcite{el2018towards}, ~\brackcite{zheng2020fused},
~\brackcite{perttunen2011distributed}, ~\brackcite{gonzalez2017learning}].\\
\indent El objetivo principal de este trabajo es proponer un modelo para la detección de baches utilizando los sensores embebidos en los
dispositivos móviles. Para lograr esto se pretende:

\begin{itemize}
	\item Estudiar el estado del arte del uso de Dispositivos Móviles para la Detección de Baches.
	\item Determinar el conjunto de sensores apropiados para la detección de baches, así como su frecuencia de muestreo.
	\item Implementar un prototipo de aplicación para la captura de los datos de los sensores, así como la señalización
		manual de baches.
	\item Proponer e implementar un modelo para la detección automática de baches.
	\item Proponer e implementar un prototipo para la visualización de los datos.
	\item Proponer una metodología para validar los resultados parciales.
\end{itemize}
