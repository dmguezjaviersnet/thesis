\chapter*{Introducción}\label{chapter:introduction}
\addcontentsline{toc}{chapter}{Introducción}

Nuestro país se ha propuesto desde hace unos años llevar a cabo un profundo proceso de informatización.
Transacciones bancarias, datos móviles, digitalización de información, acceso a información, etc. Todos
estos han sido, y aún son, temas dentro de dicho proceso con el objetivo de facilitar muchas actividades
cotidianas. En los tiempos actuales las carreteras son algo común y los vehículos que hacen uso de estas
también. No obstante, a menudo las carreteras se deterioran debido al clima o al tráfico, y en nuestro
país es algo bastante común que la mayoría de las carreteras en las grandes ciudades estén en malas condiciones.

\section*{Motivación}
En nuestro país no existe ninguna forma de ubicar de forma sencilla y accesible los tramos de carreteras que están
en mal estado. Sería muy útil poder hacer esto pues de esta forma se podría tener acceso a esta información de forma
instantánea. El acceso rápido a estos datos no solo facilitaría la reparación de estas calles, sino que también podría
servir de guía a choféres para evitar tales tramos y por tanto consecuencias ocasionadas por estas condiciones como 
accidentes y daños a la estructura de los vehículos. Además, el hecho de no tener que enviar vehículos a inspeccionar
el estado de las carreteras ya constituye un ahorro importante de recursos y de tiempo, y permitiría tomar decisiones 
con mayor rapidez sobre que tramos de carretera priorizar.

\section*{Problema}
Determinar cuando hay un bache en la carretera, y en que posición se encuentra valiéndose de un dispositivo móvil.


\section*{Objetivos}
\subsection*{General}
Proponer un modelo para la la detección de baches utilizando dispositivos móviles.

\subsection*{Específicos}
\begin{enumerate}
		\item Estudiar el estado del arte del uso de Dispositivos Móviles para la Detección de Baches
		\item Determinar el conjunto de sensores apropiados para la detección de baches así como su 
frecuencia de muestreo. 
		\item Implementar un prototipo de aplicación para la captura de los datos de los sensores así 
como la señalización manual de baches. 
		\item Proponer e implementar una modelo para la detección automática de baches.
		\item Proponer e implementar un prototipo para la visualización de los datos.
		\item Proponer una metodología para validar los resultados parciales.

\end{enumerate}

\section*{Propuesta de solución}

Nuestra propuesta de solución es una aplicación móvil que capture las señales de los sensores integrados de un dispositivo móvil, almacene
dicha información, y con esta, construir un modelo de Machine Learning que sea capaz de clasificar en tiempo real las irregularidades en la
carretera y ubicarlas en un mapa disponible para los usuarios de la aplicación.
