\chapter*{Introducción}\label{chapter:introduction}
\addcontentsline{toc}{chapter}{Introducción}

Nuestro país se ha propuesto desde hace unos años llevar a cabo un profundo proceso de informatización. Transacciones bancarias, datos móviles, digitalización de información, acceso a información, etc. Todos estos han sido, y aún son, temas dentro de dicho proceso, con el objetivo de facilitar muchas actividades cotidianas. En los tiempos actuales, las carreteras son algo común, y los vehículos que hacen uso de estas también. No obstante, a menudo, las carreteras se deterioran, y en nuestro país, es algo bastante común que la mayoría de las carreteras en las grandes ciudades estén en malas condiciones.

\section*{Motivación}
Existen varias investigaciones que proponen soluciones para detectar las irregularidades en la carretera, la mayoría utilizan el acelerómetro. Trabajos como [1] The Pothole Patrol, donde utilizan acelerómetros externos que toman muetras con una frecuencia de 380Hz y un GPS externo. Luego estos datos pasan por varios filtros (5 en este caso), cada uno de los cuales se encarga de descartar eventos no asociados a irregularidades en la carretera. O por ejemplo en [2] Real Time Pothole Detection using Android Smartphones with Accelerometers, donde proponen un método utilizando los datos del acelerómetro, para detectar las anomalías, denominado G-ZERO, cuya hipótesis es que en el intervalo de tiempo en el que ocurre una anomalía, las lecturas del acelerómetro en los 3 ejes es bastante cercana a 0 y menor que cierto valor, lo que se asocia con la caída libre del vehículo en ese pequeño intervalo de tiempo.

\section*{Problemática}
En nuestro país no existe ninguna forma de ubicar de forma sencilla y accesible los tramos de carreteras que están en mal estado. Sería muy útil poder hacer esto, pues de esta forma se podría tener acceso a esta información de forma instantánea. El acceso rápido a estos datos no solo facilitaría la reparación de estas calles, sino que también podría servir de guía a choféres para evitar tales tramos y por tanto consecuencias ocasionadas por estas condiciones.

\section*{Objetivos}
\subsection*{General}

Desarrollar una aplicación móvil que sea capaz de identificar y clasificar irregularidades en la carretera utilizando los sensores integrados del teléfono móvil.

\subsection*{Específicos}
\begin{enumerate}
		\item Analizar el estado del arte respecto a la problemática en cuestión.
		\item Recolectar información utilizando los sensores del móvil, que nos permita construir una base de datos.
		\item Clasificar los distintos tipos de irregularidades en la carretera utilizando los datos recolectados previamente mencionada. 
		\item Ubicar en un mapa las irregularidades haciendo uso de los datos del GPS almacenados en la base de datos.
		\item Crear una base de datos con información acerca de la localización de tramos en mal estado. 

\end{enumerate}

\section*{Propuesta de solución}

\section*{Estructura de la tesis}

\section*{Estado del arte}
