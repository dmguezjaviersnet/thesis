\chapter{Estado del Arte}\label{chapter:state-of-the-art}

Existen varios artículos científicos que proponen soluciones al problema de identificar cuando hubo una anomalía, algunos proponen métodos
puramente estadísticos y otros utilizan métodos de aprendizaje de máquinas para identificar patrones, mayormente en la serie temporal del
acelerómetro. También utilizan procesamiento de imágenes para poder detectar sobre el pavimento la región o las regiones que se corresponden
con un bache utilizando distintos filtros sobre la imagen.

\section{Detección en Imágenes}

\subsection{Segmentación de la imagen}
La segmentación de imágenes permite  dividir la imagen del pavimento en regiones defectuosas y no defectuosas ~\parencite{koch2011pothole}.
El primer paso sería transformar la imagen original a color en una imagen a escala de grises  ~\parencite{koch2011pothole}.

Es una de las propuestas es segmentar la  imagen en regiones defectuosas y no defectuosas ~\parencite{koch2011pothole}
Luego utilizar un proceso de  \emph{thresholding} sobre un histograma de la forma de la imagen (histogram shape-based thresholding)

Con base en las propiedades geométricas de una región defectuosa, la
la forma potencial del bache se aproxima utilizando adelgazamiento morfológico y regresión elíptica.~\parencite{koch2011pothole}


Posteriormente, se extrae la textura dentro de una posible forma defectuosa y se compara con la textura del pavimento circundante 
sin defectos para determinar si la región de interés representa una forma real.

bache.ndear el pavimento sin defectos para determinar si la región de interés representa un real bache

En las imágenes de la superficie del pavimento, la información de color, en particular
Los valores RGB no son imprescindibles a la hora de realizar la segmentación proceso con respecto a 
la detección de defectos. Por lo tanto, el primer paso es transformar imágenes en color originales en 
imágenes a escala de grises.

Para cumplir con la primera característica identificada, las regiones más oscuras que indican defectos deben separarse de la parte posterior.
suelo dentro de cada imagen de pavimento.

Después de llenar los agujeros dentro de las regiones restantes del binario
imagen B, regiones pequeñas (p. ej., grietas cortas, artefactos), regiones que

tener una forma lineal (por ejemplo, grietas largas, juntas, cortinas de bordillo) y re-
Se asumen giones, que están conectados al límite de la imagen.
Para que no queden baches y así se eliminen.


De acuerdo con la observación de los autores de que la región dentro de una olla
el agujero es principalmente granular, los llamados filtros puntuales se utilizan para crear
respuestas de filtro alto


Entre los varios métodos estadísticos ~\parencite{mednis2011real} propusieron los siguientes:\\

\begin{itemize}
	\item  \emph{\textbf {Z-Thresh}}: Toma el valor del acelerómetro en el eje Z en un instante determinado y compara su valor modular con 
	un valor \emph{threshold} prefijado, si es mayor devuelve \emph{True} indicando que hay una anomalía en ese instante, de lo contrario
	devuelve \emph{False}.\\
	\item \emph{\textbf {Z-DIFF}}: Utiliza el valor del acelerómetro en el eje Z en dos instantes de tiempo consecutivos y compara la
		diferencia modular con un valor \emph{threshold} prefijado, si es mayor devuelve \emph{True} indicando que hay una anomalía en ese
		instante, de lo contrario devuelve \emph{False}.\\
	\item \emph{\textbf {STDEV(Z)}}: Analiza la desviación estándar del valor del acelerómetro en el eje Z durante un intervalo de
		tiempo y compara con un valor \emph{threshold} prefijado, si es mayor devuelve \emph{True} indicando que hay una anomalía en ese
		instante, de lo contrario devuelve \emph{False}.\\ 
	\item \emph{\textbf {G-ZERO}}: Este método se basa en la idea de que cuando el vehículo interactúa con alguna anomalía, existe un
		instante de tiempo donde las lecturas del acelerómetro en los 3 ejes son cercanas a 0 y menores que cierto valor prefijado, lo que
		indica que el vehículo está en caída libre.\\
\end{itemize}

Los 3 primeros métodos requieren información acerca de la orientación del eje Z imaginario del dispositivo encargado de recopilar la información. 
Para capturar datos preliminares utilizaron un collar LynxNet a una frecuencia de 100Hz. Para la evaluación del sistema utilizaron 4 dispositivos
móviles (Samsung i5700, Samung Galaxy S, HTC Desire, HTC HD2), con los que recolectaron más datos y los procesaron con el objetivo de asignar los 
valores óptimos a cada uno de los \emph{threshold}s característicos de cada uno de los 4 métodos propuestos, así como el tamaño de la ventana en caso de 
\textbf {STDEV(Z)}.\\

~\textcite{mohan2008nericell} para el diseño de su estrategia se basaron en 2 ideas, cuando un vehículo interactúa con una
anomalía (un bache en este caso) las ruedas descienden en el hueco ocasionando una caída sostenida en la aceleración en el eje Z,
hasta que alcanzan el fondo del bache y en ese momento se produce un pico bien elevado de aceleración en el eje Z. A altas velocidades
ese pico al final del evento es bien prominente, sin embargo cuando se va a muy poca velocidad no se nota prácticamente, pero la caída
sostenida en la aceleración en el eje Z al entrar al bache si se mantiene, por lo que propusieron 2 métodos para detectar una anomalía
dependiendo de la velocidad $v$ a la que se desplaze el vehículo:\\

\begin{itemize}
	\item \textbf {v > 25 km/h}:  Utilizan un método idéntico al \textbf {Z-THRESH}, o sea, buscan un valor bien prominente de
		aceleración en un instante de tiempo determinado, que sea mayor que cierto \emph{threshold} prefijado.\\
	\item \textbf {v < 25 km/h}:  Utilizan un método que llamaron \textbf {Z-SUS} que busca una caída sostenida en la aceleración
		en el eje Z, por debajo de un cierto \emph{threshold} prefijado, durante cierta cantidad de tiempo.\\
\end{itemize}

Con respecto a los trabajos existentes que utilizan el aprendizaje de máquinas existen varios, ~\textcite{carlos2018evaluation} además de realizar
una comparación de los métodos existentes en su momento, crearon un método en el que utilizaron un enfoque supervisado con \emph{Support Vector Machines} 
(\textbf{SVM}) para entrenar un modelo que decidiera utilizando varios \emph{features} (basados principalmente en la aceleración en el eje Z), sí existía
o no una anomalía. \emph{Features} como la media, desviación estándar, varianza, coeficiente de variaciónn y diferencia entre valor mínimo y máximo. Otros
4 \emph{features} utilizados para otorgar un valor de confianza a 4 de los \emph{features} previamente mencionados excepto a la varianza, comparándolos con
un cierto valor \emph{threshold}. Esta idea fue tomada teniendo en cuenta una serie de artículos con respecto al tema(~\textcite{mednis2011real}), con el
objetivo de mejorar el rendimiento del clasificador. Otro de los \emph{features} es la cantidad de veces que un feature estadístico sobrepasó el valor
\emph{threshold} concebido. Completan los 12 \emph{features} utilizados la suma de los valores de confianza, así como el valor de confianza correspondiente.
Cabe desctacar que implementaron una plataforma llamada Pothole Lab con el objetivo de crear un sitio web público donde tener acceso a data sets robustos y
curados. Los datos que utilizaron fueron obtenidos de este mismo sitio y fueron capturados utilizando un Moto G Android a una frecuencia de 50Hz.\\

~\textcite{seraj2015roads} en su propuesta, incorporaron además los datos del giroscopio con el objetivo de obtener más \emph{features} que permitieran mejorar
la calidad del clasificador. Además, también realizaron un proceso para eliminar el ruido y mejorar la calidad de las señales obtenidas con los sensores de los
dispositivos utilizados para capturar los datos. Separaron las señales en ventanas de 2.5 segundos y extrajeron de ahí varios \emph{features} para el proceso
de entrenamiento. \emph{features} de dominio temporal, \emph{features} de dominio de frecuencia y \emph{features} extraídos de la transformada de wavelet son
los que conforman los vectores. Finalmente entrenaron un modelo utilizando una \textbf{SVM} con los \emph{features} que extrajeron de los datos, razón por la
cual etiquetaron sus datos utilizando vídeos y audios grabados durante la recolección de los datos. El dispositivo que utilizaron para la recolección de los
datos fue un Samsung Galaxy S2 y recopilaron datos a frecuencias de 47Hz y 93Hz.\\

~\textcite{el2018towards} como en el artículo previamente mencionado, llevaron a cabo un extenso proceso, separando las señales de los sensores por
ventanas de 1 segundo y generando vectores de \emph{features} bien extensos en cada una de estas ventanas. Además llevaron a cabo un proceso de eliminación
de ruido de las señales probando varios métodos conocidos en el campo del procesamiento de señales digitales como la transformada de Fourier o la transformada de
Fourier por ventanas discretas (\textbf{WDFT} por sus siglas en inglés), pero finalmente utilizaron transformadas de wavelets para purificar la señales y obtener
datos más confiables con los que llevar a cabo el proceso de entrenamiento del modelo de \emph{Machine Learning}. Se decantaron por un enfoque supervisado, por
lo que etiquetaron de antemano los datos utilizando vídeos capturados durante el mismo proceso de recolección de datos, métodos como \emph{K-Nearest Neighboors}
(\textbf{KNN}), \emph{\textbf{Decision Trees}}, \textbf{SVM} y \emph{ensembles} de clasificadores fueron probados, obteniendo los mejores resultados con las
\textbf{SVM} y los \emph{bagged decision trees}. Utilizaron \emph{features} estadísticos, de dominio temporal, de dominio de frecuencia y de dominio
frecuencia-temporal. Entre los \emph{features} que utilizaron están media, mediana, moda, desviación estándar, varianza, rango entre cuartiles, media de la
frecuencia, la mediana de la frecuencia, etc.\\

% //\\// Pendiente //\\//
% - Añadir detalles sobre el uso del GPS.
% - Añadir detalles sobre si es multi-class classification o binary-classification
% - Añadir detalles sobre el uso de distintos vehículos.
% - Añadir detalles sobre la ubicación de los dispositivos en el vehiculo.

El algoritmo para la detección de anomalías en la carretera propuesto por ~\textcite{eriksson2008pothole} se basa en que condiciones anómalas de la carretera
son reflejadas en características de los datos de aceleración. El problema para identificar baches a partir de lecturas del acelerómetro es bastante
complelejo debido a la amplia variación en las condiciones de la carretera y el comportamiento del conductor. La mayoría de las anomalías pueden ser
caracterizadas como eventos de alta energía en la señal de la aceleración, por si sola la energía de la señal no es suficiente como un criterio de
detección debido a que muchos eventos de alta energía pueden no considerarse anomalías en la carretera. Muchos accesorios de la carretera como
los cruces de trenes, las juntas de expansión, los eventos de alta energía que pueden ser causados por los pasajeros cuando le dan un fuerte
tirón a la puerta del vehículo o si el el conductor frena de manera repentina. Los datos recopilados son divididos en 256 ventanas pues
los eventos que interesan son generalmente eventos de corta duración. Una serie de filtros de procesamientos de señales son aplicados
al  pedazo continuo de ventanas, donde cada filtro está diseñado para rechazar uno o más de un evento que no sea bache.
Las etapas de filtrado se dividen en 5 y son las siguientes:

\begin{itemize}
	\item \emph{\textbf {Velocidad}}: Las ventanas donde el carro no se está moviendo o se está moviendo a muy poca velocidad son ignoradas. 
		De esta forma se rechazan eventos como un tirón en la puerta cuando se baja el pasajero del vehículo.\\
	\item \emph{\textbf {High-pass}}: Los filtros de \emph{high-pass} eliminan las componentes de baja frecuencia de la señal de aceleración en los ejes z y x, 
		que pueden indicar giros, frenazos, etc.\\
	\item  \emph{\textbf {Z-peak}}: Los picos de aceleración en el eje z es la principal característica de las anomalías significativsas de la carretera.
		Este filtro rechaza todas las ventanas con un pico (absoluto) en la aceleración menor que un threshold $t_z$.\\
	\item \emph{\textbf {xz-ratio}}: La aceleración en el eje x puede ayudar a identificar anomalías que abarcan el ancho de la carretera y por lo tanto 
		impactan ambos lados del carro de igual manera.  Asumiendo que el bache solo impacta un lado del carro, un verdadero evento de bache con un
		un gran pico de aceleración en el eje z debería producir un pico significante en el eje x dentro de alguna ventana.\\
	\item \emph{\textbf {speed vs ratio }}:	A altas velocidades, incluso pequeñas anomalías en la carretera pueden crear altos picos en las lecturas 
		de aceleración. Este filtro rechaza ventanas donde el pico de aceleración en el eje z es menor que un factor $t_s$ veces la velocidad a la que 
		va el vehículo.
\end{itemize}

Dado que la frecuencia de captura de la señal del acelerómetro se hace mucho más seguido que la del GPS, proponen estimar la localización del vehículo
cuando ocurra la lectura $l_i$ del acelerómetro usando interpolación lineal entre lecturas de GPS. A pesar que ciertos tipos de anomalías pueden producir
alta energía, no siempre que se produzca esto representa una carretera en mal estado. Los cruces de ferrocarriles, reductores de velocidad y otros
equipamientos bien conocidos embebidos en la carretera pueden producir una alta energía y no se consideran anomalías de la carretera. Para esto proponen
tener una lista negra con los localizaciones de las anomalías que caigan en alguno de estos equipamientos embebidos en la carretera cuya información se
puede obtener de y luego remover los mismos. Debido a que el error en la medición del GPS (alrededor de 5m a la redonda como media) es significativamente
mayor que el tamaño de un bache común, la ausencia de una detección en una localización particular no siempre es un indicativo de que no haya anomalías en
esta []. No es posible determinar puramente por la localización del GPS en el momento en que las ruedas del vehículo hacen contacto con alguna anomalía de
la carretera. Los conductores usuales intentan evitar los baches, por que la probabilidad de detectar una anomalía en la carretera es menor que la que se
debe de esperar de una distribución sin sesgo del área que se intenta mapear.\\

~\textcite{cong2013applying} para la extracción de \emph{features} descomponen la señal usando \textbf{WPD}(\emph{Wavelet Packet Decomposition}) usando
sucesivamente filtros de \emph{low-pass} y de \emph{high-pass}. Donde \textbf{WPD} es llevado a cabo aplicando de forma iterativa filtros espejos en
cuadratura y seguido por submuestreo. La selección de \emph{features} basadas en \emph{Machine Learning} puede ser asistida para reducir la demanda
computacional para la clasificiación. La selección de \emph{features} está diseñada para encontrar las \emph{features} que hacen una mejor discriminación
de los baches y los segmentos normales en estudio. Para la selección de \emph{features} se probaron con cuatro métodos: \textbf{BS}(\emph{Backward Selection}),
\textbf{FS} (\emph{Forward Selection}),  y \textbf{PCA} usando un número diferente de \emph{features} seleccionadas. \textbf{PCA} fue el que
arrojó mejores resultados cuando el número de \emph{features} es mayor que 5, mientras \textbf{FS}(\emph{Forward Selection}) es mejor cuando el número de
\emph{features} es mayor que 2 y menor que 6.\\
Para la clasificicación usaron \textbf{SVM} one-class classification con kernel RBF con parámetros  $\nu = 0.01$  $\gamma = 0.00002$. \textbf{SVM} fué
entrenado con el 70 \%  de los datos (1234 segmentos) y el conjunto de entrenamiento fue escogido de manera aleatoria. El resto de los datos  (530 segmentos)
y en todos los 21 segmentos de anomalías fueron usados para los tests de precisión del modelo \textbf{SVM} construido.\\

~\textcite{kulkarni2014pothole} proponen un sistema que detecta los baches, registra su ubicación, un crea un documento que
puede utilizarse para cargarlo en un servidor centralizado o enviarlo  a las autoridades competentes inmediatamente.  Cuando el usuario inicia su viaje,
lanza la aplicación Android de detección de baches. La aplicación, que tiene como complemento del algoritmo en funcionamiento, detecta los baches en las
carreteras mientras el usuario está conduciendo. Supervisa los cambios en la aceleración. La aplicación añáde la hora actual, las coordenadas geográficas
y las estadísticas de baches al registro de eventos. Cuando el usuario  finalice el recorrido pulsa "Stop"  y se le presenta el registro de eventos. Este
registro debe mantenerse en la base de datos.  El algoritmo que proponen es el siguiente:\\\\
\noindent

\begin{itemize}
	\item  Un filtro de \emph{high-pass} para remover las componentes de baja frecuencia de la señal de aceleración en los eje x y z.  El filtro de \emph{high-pass}
		elimina el desplazamiento de la gravedad. El alpha usado es de 0.8 \\\\
	\begin{align*}
		\alpha = 0.8 \\
		gravity_{x} = \alpha * gravity_x + (1-\alpha) *event.values_{x} \notag\\
		gravity_{y} = \alpha * gravity_y + (1-\alpha) *event.values_{y} \notag\\
		gravity_{z} = \alpha * gravity_z + (1-\alpha) *event.values_{z} \notag\\	
	\end{align*}
			
	Luego el effecto de \emph{high-pass} para la eliminación de las componentes de baja frecuencia
	
	\begin{align*}
		acceleration_{x} =  event.values_{x} - gravity_{x}\\
		acceleration_{y} = event.values_{y} - gravity_{y}\\
		acceleration_{z} = event.values_{z} - gravity_{z}
	\end{align*}

	\item Los picos de aceleración en el eje Z es una de las características principales de las anomalías en la carretera. Este filtro rechaza todas 
	las ventanas con un pico de aceleración en la componente z menor menor que un \emph{threshold} \textbf{tz} ( o sea rechaza la lectura si $accel_z < tz$ ).
	\item Un verdadero evento de bache con una larga aceleración en la componente z debe producir un pico significativo en el eje x. Este filtro rechaza 
	todas las ventanas con un pico de aceleración en la componente z menor que  el producto del \emph{threshold} en el eje x \textbf{tx} por  el pico de aceleración (o sea 
	si tx es el \emph{threshold} en el eje x rechazaría las lecturas que $accel_x < tx * accel_z$).
	\item A altas velocidades pequeñas anomalías pueden crear altos picos en las lecturas de aceleración. Este filtro rechaza las ventans donde los picos de 
	aceleración en el eje z son menores que un factor ts veces la velocidad a la que se viaja(O sea asumiendo que el factor de velocidad es ts, y \emph{speed} la velocidad actual 
	a la que viaje el vehículo rechaza las lecturas que cumplan $accel_z < ts * speed$). 
	\item  Si todas las condiciones anteriores se cumplen entonces se considera un bache o en caso contrario no.
	\item  Se usa una red neural para mejorar la eficiencia y precisión de la detección de baches. Los parámetros de la red neural usados son los siguientes
	\begin{enumerate}
		\item Número de capas de entrada: 3
		\item Input 1 : aceleración en el eje x
		\item Input 2 : aceleración en el eje x
		\item Input 3 : aceleración en el eje x
		\item Número de neuronas ocultas: 6
		\item Número de capas de salidas: 1
		\item Output 1 : Decidir si es un bache
		\item Función de activación: sigmoidal
		\item Algoritmo usado: \emph{backpropagation}
	\end{enumerate}
	
\end{itemize}

~\textcite{zheng2020fused} se refieren a que la mayoría de los artículos con respecto al tema no toman en consideración el hecho que la gran mayoría de la
carreteras que existe en el mundo no posee anomalías, y que la aplicación de técnicas de \emph{Machine Learning} utizando una ventana deslizante a ciegas puede
disminuir considerablemente la precisión y la rapidez del proceso de entrenamiento. Para esto plantean que una anomalía comienza con una señal normal
y luego ocurre un pico de aceleración en el eje Z que excede un \emph{threshold} superior y luego cae y excede un \emph{threshold} inferior o viceversa,
y finalmente la señal vuelve a estabilizarse entre esos dos \emph{threshold}s. Ese intervalo es el que consideran como el intervalo candidato en el que
puede haber una anomalía, y para encontrarlo utilizan primero el método heurístico de establecer un \emph{threshold} superior e inferior en la aceleración
y buscan el primer instante donde la aceleración sobrepasa alguno de los \emph{threshold}s y el último instante, y utilizan un método para encontrar los
instantes de tiempo justo antes de que ocurriera el primer evento y el instante de tiempo justo después de que ocurriera el último evento, de esta forma
logran construir una ventana de tamaño dinámico que contiene toda la información acerca de la anomalía, obteniendo como resultado un conjunto de ventanas
con posibles anomalías en la serie temporal. Luego crean un modelo con un \emph{Random forest} utilizando algunos \emph{features} estadísticos de cada una
de las ventanas para filtrar ventanas que no constituyen anomalías reales, también proponen tratar al vehículo como un modelo de vibración de un nivel de
libertad para lo cual diseñan varios experimentos. Como resultado de esto llegaron a la conclusión de que la varianza de la aceleración en el eje Z cuando
el vehículo interactúa con una anomalía tendrá una relación casi lineal con la profundidad o altura de la anomalía. Utilizando \textbf{KNN} y \textbf{DTW}
(\emph{Dynamic Time Warping} que se utiliza para determinar similitud entre 2 series temporales que puedan ser obtenidas a distintas velocidades), es que
llevan a cabo el proceso de identificar los tipos de anomalías. Los autores comparan su propuesta con otras hechas en los últimos 3 años obteniendo un mejor
\emph{F1 score} al identificar los 3 tipos de anomalías que consideraron en cada uno de los 3 \emph{datasets} que probaron.\\

~\textcite{perttunen2011distributed} explican que los sensores del teléfono móvil son reflejados en las señales de dos maneras: Primero, la señal del
\textbf{GPS} en el teléfono usado tenía muchco ruido. Segundo, las mediciones del \textbf{GPS} y de la aceleración fueron contaminadas por ráfagas,
que son mediciones registradas con la misma marca de tiempo. A continuación, se realizó el rechazo de valores atípicos de GPS y se aplicó el filtro de
Kalman a la latitud y la longitud para reducir aún más el ruido. Después la velocidad será estimamda por cada par consecutivo de latitud y longitud.
La señal resultante de velocidad fué sobremuestrada y filtrada usando límites físicos razonables para la aceleración del vehiculo. Cuando las estimaciones
de velocidad parecían lo suficientemente suaves, la señal estaba a menudo contaminada por una gran latencia en comparación con la estimación de velocidad
original (ruidosa) y la señal no llegaba a cero en las paradas de los vehículos. Para aliviar este problema se usan dos correcciones. En primer lugar, se
elimina una latencia determinada visualmente de la estimación de velocidad filtrada. En segundo lugar, se aplica una fusión muy sencilla de la aceleración
y la señal de \textbf{GPS}: la varianza de la norma de la aceleración se calculó para la señal de aceleración y se examinó visualmente. Mediante un simple
\emph{threshold} fueron capaces de detectar un segmento de la señal, donde la velocidad del vehículo era cero, o muy cercana a cero. A continuación,
establecieron los segmentos correspondientes de la señal de velocidad en cero y suavizaron la señal utilizando un filtro de Kalman. Los datos fueron
seperados usando una ventana deslizante. Experimentaron con pedazos desde 0.5 segundos hasta 2 segundos, esta escala se consideró aceptable para la tarea de
reconocimiento de anomalías, ya que la duración media de las anomalías era alrededor de 2 segundos. Por cada ventana, se determinaba el por ciento de la
ventana cubierta por anomalías (uno o más juntos). La extracción de características se realiza utilizando ventanas deslizantes de 2 segundos de longitud,
con un deslizamiento de 0.5 segundos. Varias características fueron extraídas de la señal de aceleración: la desviación estándar, la media, la varianza,
\emph{peok-to-peak}, \emph{signal magnitude area}, \emph{3-order autoregressive coeficcients}, \emph{til angles}, la raíz de la desviación cuadrática media.
Los valores absolutos de la correlación de señales entre todas las dimensiones son usados también, ya que se observó visualmente muchas veces todas las
señales de aceleración mostraban similar formas de onda en los segmentos de la anomalía. Se utilizaron características basadas en \emph{Fast Fourier
Transformation(FFT)} para incorporar información de frecuencias específicas. Esto se basó en la suposición de que los baches producirían componentes de
menor frecuencia en comparación a la vibración que se origina en el motor y en la superficie normal de la carretera. La energía \textbf{FFT} se extrajo de
17 bandas de frecuencia para cada dirección de la aceleración y los coeficientes cepstrales de frecuencia de mel en 4 bandas. Se utilizó el algoritmo de
selección de \emph{features} \emph{Backward Selection} para seleccionar el conjunto de \emph{features} óptimo tanto para los que eliminaron la dependecia
de velocidad como para los que no. Usaron un método para remover la dependecia de la velocidad en los \emph{features}. Para clasificar las ventanas, que
representan pequeños segmentos de la carratera, se utilizó SVM con kernel RBF. El clasificador con el mejor g-means medio luego de varias corridas fue
seleccionado. Evaluaron 49 combinaciones de parámetros con \textbf{SVM} usando función de kernal radial (RBF), correspondientes a una rejilla de valores
de los parámetros $\gamma$ y C. Presentaron además un \emph{framework} de visualización para los resultados, parar habilitar la inspección visual de,
por ejemplo, los ejemplos cercanos al borde de las clases.\\

~\textcite{pawar2020efficient} utilizan un dispositivo móvil ubicado en el parabrisas del vehículo para capturar las señales del acelerómetro y el
giroscopio. Con esto construyen un vector de 24 \emph{features} entre los cuales están los valores máximo, mínimo, la desviación estándar y la media
de los 3 ejes del acelerómetro así como los del giroscopio, luego estandarizaron los datos para disminuir el efecto de los \emph{outliers} en los
\emph{features} y permite percibir mejor en un gráfico los picos en las lecturas causados por los baches. Finalmente proponen como modelo una red
neuronal con 2 capas ocultas y 1 capa de salida, cada una de las capas ocultas compuesta de una capa densa con función de activación ReLU, permitiendo
\\emph{backpropagation} de forma más rápida. Adicionalmente cuenta con una capa de \emph{dropout} con el objetivo de prevenir el \emph{overfitting},
asignando 0 como valor de forma aleatoria a un porciento de las entradas luego de cada actualización. La capa de salida tiene como función de activación
la sigmoidal. La función de pérdida es \emph{binary cross entropy} y la red neuronal es entrenada durante 140 épocas con un \emph{batch size} de 3.
Compararon su propuesta con otras similares pero modificando los hiperparámetros e incorporando un modelo conocido como \textbf{SMOTE}(\emph{Synthetic
Minority OverSampling Technique}) con el objetivo de contrarrestar el gran desbalance entre clases que hay en este problema en particular(bache y no
bache en este caso).\\ ~\textcite{gonzalez2017learning} realizaron un riguroso proceso de experimentación con respecto al tipo y total de vehículos con
un total de 12, entre ellos camiones incluidos, además probaron con distintas posiciones del móvil dentro del vehículo. Para la recolección de datos 2
personas iban en el vehículo, uno de ellos encargado de etiquetar las anomalías de la siguiente forma, cuando se van acercando a la anomalía ponen el
dispositivo a recolectar los datos y una vez pasan la anomalía detienen la recolección de datos y asignan una etiqueta a dicha serie temporal. Su
propuesta consiste en una forma distinta de procesar los datos capturados con el acelerómetro del dispositivo móvil, que es utilizando lo que se conoce
como \emph{Bag of Words representation}, donde los datos son representados utilizando un histograma que refleja la cantidad de veces que se repite una
palabra en un documento determinado. Esta representación se ha utilizado en áreas como visión por computadoras, procesamiento del habla y procesamiento
de series temporales como es el caso de este problema en particular. El objetivo es crear un diccionario de palabras clave que describa la serie temporal
por fragmentos (un tamaño prefijado) y luego representar en un histograma la frecuencia con que aparece cada palabra clave en una serie temporal dada.
Para crear esta lista de palabras clave separan en una cantidad de tramos prefijada las series temporales de los datos recolectados y etiquetados, con
datos de estos tramos como las lecturas, la media, la varianza, y los valores máximo y mínimo corren \emph{k-means} con todos los datos de una clase para
cada una de las clases, con el objetivo de agrupar los tramos similares en \emph{clusters} y obtener $k$ centroides, cada uno de estos lo consideran una
palabra clave perteneciente a una de las clases en cuestión. De esta forma cada clase tiene $k$ centroides que son sus palabras clave y así se obtiene 
un diccionario de clase:palabras clave. Luego para construir el vector de \emph{features} segmentan la señal por tramos y comparan cada tramo con cada
uno de las palabras clave en el diccionario utilizando distancia euclideana y asignando a cada tramo la palabra clave con menor distancia, de esta forma
cada tramo se convierte en una palabra clave y es convertido a un histograma donde cada característica constituye la cantidad de veces que se repite cada
palabra clave en esa serie temporal, y es este vector el que entregan al clasificador. Probaron con una gran variedad de clasificadores redes neuronales
artificiales(\textbf{ANN}), \textbf{KNN}, \emph{Naive Bayes}(\textbf{NB}), \emph{Random Forest} (\textbf{RF}), \emph{Decision Trees}(\textbf{DT}),
\textbf{SVM}, y \emph{Kernel Ridge}(\textbf{KR}).\\
