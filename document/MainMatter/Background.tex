\chapter{Estado del Arte}\label{chapter:state-of-the-art}
En la literatura existen varios autores que proponen soluciones al problema de identificar cuando hubo una anomalía en la carretera utilizando
diferentes enfoques. Algunos plantean resolver el problema intentando identificar en imágenes las regiones que puedan ser candidatas a ser un 
bache, utilizando  técnicas de procesamiento de imágenes o de aprendizaje de máquinas. Otros proponen resolverlo utilizando métodos heurísticos
o de aprendizaje de máquinas, que hacen uso de la serie temporal que es posible generar a partir del muestreo de sensores como el acelerómetro. 

\section{Detección de baches en imágenes}
Varios autores en la literatura se han propuesto identificar baches en imágenes. Algunos de estos hacen se van por la vía del procesamiento 
de imágenes digitales. La mayoría, proponen como pasos fundamentales un preprocesamiento de la imagen utilizando principalmente 
un proceso de segmentación. Luego se encargan de busar la región de interés que representa los posibles baches en la imagen. Y por último,
tratan identificar si esa región de interés es un efecto un bache o no. 

\subsection{Preprocesamiento de la imagen}
La reducción del tamaño de una imagen  disminuye la complejidad computacional en su procesamiento ~\brackcite{jo2015pothole}. 
El tamaño de una imagen no afecta la calidad de detección de un bache ~\brackcite{akagic2017pothole}. Una de las propuestas 
es decrementar el tamaño de la imagen $\alpha$ veces \brackcite{akagic2017pothole}. \\
La segmentación de imágenes detecta información relevante, y es el primer paso en el análisis de imágenes y patrones
de reconocimiento ~\brackcite{buza2013pothole}.
La segmentación de imágenes permite  dividir la imagen del pavimento en regiones defectuosas y no defectuosas ~\brackcite{koch2011pothole}.
El paso de segmentación consiste en separar la región de un bache y de la región de fondo, transformando la imagen a color original 
en una imagen binaria utilizando el histograma de una imagen de entrada ~\brackcite{ryu2015image}.
En las imágenes sobre la superficie del pavimento la información en color (valores en RGB) no son imprescindibles a la hora de realizar 
el proceso de segmentación con respecto a la detección de defectos. Por lo tanto se  transforma la imagen original a color en una 
imagen a escala de grises  ~\brackcite{koch2011pothole, jo2015pothole}. Las regiones más oscuras que indican defectos deben separarse del fondo 
dentro de cada imagen del pavimento ~\brackcite{koch2011pothole}. Para este propósito, se puede usar el algoritmo de 
\emph{histogram shape-based thresholding} ~\brackcite{koch2011pothole,ryu2015image}, el algoritmo de máxima entropía, o el método de Otsu~\brackcite{jo2015pothole,buza2013pothole}, 
los cuales permiten separar el fondo de la imagen ~\brackcite{koch2011pothole, ryu2015image} mediante la transformación en una imagen binaria \brackcite{koch2011pothole, jo2015pothole,ryu2015image}. 
En el \emph{histogram shape-based thresholding} el valor T de umbral se determina como el valor de intensidad del punto $P_T(T, h(T))$ que tiene la distancia máxima a la recta 
que intersecta con el origen del histograma y el punto que indica la máxima intensidad $P_{max}$ ~\brackcite{koch2011pothole}. 
Basado en este umbral T se convierte la imagen mejorada $I_{m}$ en una imagen binaria B que contiene el área defectuosa ~\brackcite{koch2011pothole}.
Por otra parte, el método de Otsu separa las regiones oscuras del fondo como sigue:

$$B_{i,j}=\left\{
    \begin{array}{c l}
        1, & \text{si } I_{i,j} < T_{b}\\
        0, & \text{en otro caso}\\
    \end{array}
\right.$$
donde $B_i,j$ es el \emph{pixel}(i,j) en la imagen binaria, $I_{i,j}$ representa el correspondiente \emph{pixel}(i,j) en la imagen de entrada, y
$T_b$ es el valor de umbral del método de Otsu \brackcite{jo2015pothole}.

\subsection{Extracción de candidatos a bache}
La región de interés es el área en la que los baches pueden ser encontrados, es decir el pavimento asfáltico ~\brackcite{akagic2017pothole}.

Akagic et al. ~\brackcite{akagic2017pothole} proponen para encontrar dicha área  primero forman una nueva matriz que contiene las posiciones 
de los \emph{pixels} de todos los \emph{pixels} blancos de la imagen. Segundo, seleccionan un número dinámico de \emph{pixels}, basados 
en la desviación estándar de la imagen para formar lo que ellos denominan \emph{first level seed points}. Luego forman  los 
\emph{second level seed points} basados en los valores de los \emph{pixels} vecinos del primer nivel. La región de
interés se extrae en función de la posición de estos puntos, mientras que los valores de los \emph{pixels} del resto de la región se cambia a
0 ~\brackcite{akagic2017pothole}.

Jo et al. ~\brackcite{jo2015pothole} proponen realizar la extracción de candidatos mediantes los siguientes pasos: segmentación
de línea, detección de carril, selección de la región de interés y agrupación de línea. La segmentación de línea implica investigar la conectividad 
de línea entre los \emph{pixels}. A continuación, se detecta el carril para eliminar el ruido restante. Solo se utiliza la superficie de la carretera dentro de las dos marcas de 
carril, y todos los valores que queden fuera de esta área son eliminados. Por lo general, las marcas de carriles en una carretera son más brillantes que el fondo.
Así, se pueden separar las marcas de los carriles del fondo mediante una  mejora del brillo y el \emph{thresholding} ~\brackcite{jo2015pothole}. A continuación, los 
puntos candidatos para marcas de carriles son seleccionados mediante escaneo de líneas ~\brackcite{jo2015pothole}. Los puntos candidatos seleccionados a ser marcas de carril 
se conectan mediante la combinación de puntos cercanos. Para carreteras sin marca de carril, y siempre que no se puedan detectar carriles, utilizan 
un método de detección estático de marcas de carril. Las marcas de carril pueden ser generadas usando el \emph{vanishing point} (un \emph{vanishing point} es un punto 
en el plano de una imagen de un dibujo en perspectiva donde las proyecciones en perspectiva bidimensional de líneas paralelas entre sí en el espacio 
tridimensional parecen converger) ~\brackcite{jo2015pothole}. La mayoría de los carriles tienen dos marcas de carril (una hacia el lado derecho 
del vehículo, y otra hacia su lado izquierdo). Si una de las dos marcas no es detectada, es reemplazada por una marca de carril estática. Solo 
se reemplaza la marca que no fue detectada. Es decir, cuando una marca del carril izquierdo no se detectó, solo esa línea se reemplaza por una marca 
de carril estática. El área entre dos marcas de carril detectadas es seleccionada como la región de interés. La agrupación 
de línea se aplica dentro de la región de interés. La agrupación de línea también remueve el ruido creado por el proceso de \emph{thresholding}.

Ryu et al. ~\brackcite{ryu2015image} primero proponen usar un \emph{median filter}. La idea principal del \emph{median filter} es recorrer la señal 
entrada por entrada, reemplazando cada entrada con la mediana de las entradas vecinas. Este filtro es utilizado para eliminar ruido como grietas o manchas. 
Se probaron filtros de 3x3, 7x7 y 9x9 y el filtro 9x9 mostró el mejor desempeño entre los tres. A continuación, se restauran los contornos dañados de las 
regiones del objeto, y las piezas pequeñas se eliminan mediante la operación de \emph{closing}(dilatación y erosión) de un filtro morfológico. Después de la operación 
de \emph{closing}, las regiones candidatas a bache se extraen utilizando características como el tamaño, la compacidad, la elipticidad
y la linealidad. Además, el refinamiento de las regiones candidatas es necesario para detectar las regiones correctas que son baches después de obtener las regiones 
candidatas. Los candidatos iniciales obtenidos usando características pueden no representar el área total del bache. De esta forma, el refinamiento de las 
regiones candidatas usando características como la compacidad, el punto central, y el \emph{convex hull} son necesarios para decicidir 
si varias regiones candidatas incompletas tales como, sombras, manchas y parches son baches o no.

Buza et al. ~\brackcite{buza2013pothole} proponen después de la segmentación, eliminar todas las formas lineales  y las regiones y formas 
menores que un valor $\delta$ en la imagen. La medida para eliminar toda estas regiones lineales se define por su excentricidad. Todas 
las formas que están conectadas al límite de la imagen también son eliminadas. Mediante un algoritmo de agrupamiento spectral extraen 
las regiones para representar formas en una imagen a escala de grises. Después de aplicar  el agrupamiento en la imagen,  
se notó como resultado que había un color que tenía mayor densidad en la región del bache que otros.

Koch et al. ~\brackcite{koch2011pothole} logran distinguir entre regiones que podrían representar la forma de un bache y 
regiones que podrían representar un bache completo. Una región es candidata a representar la forma de un bache si 
el centroide de la región ($P_{cent}$) no se encuentra dentro de la región, o si la excentricidad excede un valor de umbral 
máximo ($\varepsilon_{max}$) y la relación entre la longitud del eje mayor ($l_{max}$) y el tamaño de la región($s$) excede
un valor de umbral máximo ($r_{max}$). Para reducir la región a su trazo  mínimamente conectado usan un proceso
de adelgazamiento morfológico. Los puntos de ramificación del trazo son identificados y conectados en orden para 
determinar el camino principal de la región. Finalmente, teniendo como base el camino principal del trazo 
se utiliza regresión elíptica para aproximar una elipse. Con esto, se logra extraer la forma de la región de un 
candidato a bache.

\subsection{Identificación de un bache entre los candidatos}
Una vez que se tienen identificados los candidatos a ser baches, en la literatura proponen varias vías para  
identificar cuales realmente lo son.

Jo et al. ~\brackcite{jo2015pothole} emplean un método de detección en cascada para distinguir
entre baches verdaderos y falsos. Primero, al filtral con valores de umbral establecidos para el ancho, la
altura y el área, se logra remover grietas menores en la superficie y otros objetos similares. Luego, 
se utiliza un umbral de la varianza para distinguir los baches de objetos similares, como parches de 
la carretera y sombras. Los baches tienen una superficie gruesa, mientras que las regiones sombreadas 
tienen una superficie plana. Debido a esto, los baches tienen un valor de varianza alto, y las regiones 
de sombra tienen un varianza baja. El paso final para la detección en cascada es el seguimiento de 
la trayectoria para eliminar los objetos similares causados por otros vehículos en movimiento. 
Cuando un vehículo se mueve hacia la región de interés, las regiones oscuras del vehículo
corren el riesgo de ser detectadas como baches.

Akagic et al. ~\brackcite{akagic2017pothole} para la identificación de un bache proponen un método de 3 fases:
el método imágenes recortadas y umbralización de Otsu, la eliminación de formas de contorno y la detección 
de regiones de baches. La primera fase está basada en los trabajos realizados por  Buza et al. ~\brackcite{buza2013pothole}.
La primera diferencia que añaden es que la imagen de entrada en el primer paso es ahora el resultado del paso de extracción 
de la región de interés. La segunda diferencia es usan solo el paso para la eliminación de todas las formas lineales y de contorno de la imagen del
segundo paso del método anterior.


Koch et al. ~\brackcite{koch2011pothole} tratan de describir la textura de la superficie dentro de un candidato a bache y compararla con la textura de la 
región circundante. Para describir la textura de una región se propone el uso de la desviación estándar de los valores de intensidad del nivel 
de gris como una medida estadística para describir la textura  tanto de la región interna como de la región externa. A la imagen en escala de 
grises se le aplican filtros para enfatizar las características estructurales de la textura. La región exterior $R_o$ representa la parte intacta y 
saludable de la imagen del pavimento. Para determinar la región exterior se propone un  procedimiento de tres pasos. Primero, 
se aplica dilatación morfológica sobre la imagen binaria segmentada que contiene las supuestas áreas defectuosas. Segundo, 
se combina la imagen binaria con la región interior y se obtiene como resultado la región defectuosa total. Finalmente, 
La región exterior se determina como el complemento de la región defectuosa total. Basado en la respuesta de los filtros dentro de las regiones 
interior y exterior, los valores de desviación estándar correspondientes son determinados y utilizados para establecer dos vectores de características 
$f_i$ y $f_o$, que describen la textura de las dichas regiones respectivamente. A continuación, se calculan y comparan las longitudes de los vectores de 
características $f_i$ y $f_o$. Si $|f_o|$  es menor que  $|f_i|$, significa que la textura dentro de la región interior es más gruesa y granulada que en 
la región exterior. Si esto pasa, entoces la región interior se corresponde con un bache.


Ryu et al. ~\brackcite{ryu2015image} dan una propuesta para decidir si las regiones candidatas refinadas son baches o no. Para este fin, 
comparan las regiones candidatas con el fondo, usando características como la desviación estándar y el histograma. En particular, hacen 
uso de la característica del histograma mediante un \emph{ordered histogram intersection}(\textbf{OHI}). \textbf{OHI} es un método para 
medir el grado de similitud entre las regiones de una imagen. Mediante el uso del \textbf{OHI} es posible distinguir manchas, luces y 
sombras y evitar la detección errónea de baches. Para establecer que una región candidata refinada se considere que no es un bache 
determinaron dos posibles condiciones. La primera, si la desviación estándar de la región candidata es menor que un umbral establecido.
La segunda, si el \textbf{OHI} de los \emph{pixels} entre la región candidata y la región de fondo   es cercano a 1 y el \textbf{OHI}
de los valores usando la operación de Sobel es cercano a 1. Si la región candidata refininada se considera que no es un bache,
es porque es similar a la región del fondo. En caso contrario, la región candidata refinidada sería un bache.

Buza et al. ~\brackcite{buza2013pothole} después del resultado de extración de agrupamientos aplican un proceso de selección de semillas.
Esto marcará el área para la identificación de la región del bache. El próximo paso es la extracción del área vertical en la imagen agrupada.
Después de esta operación se tienen las regiones definidas verticalmentepor un bache. La operación vertical previa se utiliza como base para 
la extracción de las regiones definidas horizontalmente por un bache. El proceso de identificación de baches se completa trazando líneas verticales 
y horizontales entre los puntos indentificados para realizar la extracción de la superficie del bache.


\subsection{Detección usando \emph{Deep Learning}}
 
Varios autores dan solución  al  problema de detectar imágenes de baches en la carretera mediante el uso de una red neuronal convolucional(\textbf{CNN}).


% Para determinar la región exterior $R_o$ se propone el siguiente procedimiento:

% \begin{enumerate}
% 	\item la dilatación morfológica se aplica sobre la imagen binaria segmentada que contiene
% 	las supuestas áreas defectuosas
% 	\item Se combina la imagen binaria con la región interior $R_i$ y se obtiene como resultado la región defectuosa total.
% 	\item La región exterior $R_i$ se determina como el complemento de la región defectuosa total
% \end{enumerate}

% Basado en la respuesta de los filtros dentro de las regiones $R_i$ y $R_o$ los valores de desviación estándar correspondientes son
% determinados y utilizados para establecer dos vectores de características $f_i$ y $f_o$, que describen la textura de las regiones $R_i$ 
% y $R_o$ respectivamente. A continuación, se calculan y comparan las longitudes de los vectores de características $f_i$ y $f_o$.
% Si $|f_o|$  es menor que  $|f_i|$, significa que la textura dentro de $R_i$ es más gruesa y granulada que en el interior de $R_o$, se asume 
% que la región $R_i$ es la región interior de un bache ~\brackcite{koch2011pothole}.

% De acuerdo con la observación de los autores de que la región dentro de un bache es en su mayoría granular, 
% los llamados \emph{spot filters} se utilizan para crear respuestas de filtro alto ~\brackcite{koch2011pothole}


	% \subsection{Extracción de la forma de una imagen(\emph{Shape extraction})}
	% % El tamaño absoluto de una región de interés R es dependiente de de la resolución de una imagen, mientras 
	% % la linealidad de la forma de una región se puede definir por su excentricidad $\epsilon$. Se considera que una región es lineal 
	% % si su excentricidad es mayor que 0.99. Para extraer la forma de los próximos candidatos a regiones defectuosas se determina ~\brackcite{koch2011pothole}:
	% Dada las siguientes propiedades de una región:\\

	% \begin{itemize}
	% 	\item $\varepsilon$: La excentricidad que permite definir la linealidad de una región(Se considera que una región es lineal si su excentricidad
	% 		es mayor que 0.99)
	% 	\item $l_{max}$: la longitud del eje mayor 
	% 	\item $P_{cent}$: La posición del centroide ($P_cent$)
	% 	\item $\alpha$: La orientación del ángulo
	% 	\item R : tamaño absoluto de una región de interés dependiente de la resolución de la imagen
	% 	\item $\varepsilon_{max}$: umbral de excentricidad máxima
	% 	\item s : tamaño de la región
	% 	\item $r_{max}$: umbral de valor máximo que puede alcanzar la relación la longitud del eje mayor y el tamaño de la región($\frac{l_{max}
	% 		}{s}$) \end{itemize} 

	% Basado en estas propiedades se distingue entre regiones que podrían representar la sombra de un bache y regiones que podrían representar un bache
	% completo. Una región es candidata a representar la sombra de un un bache si el centroide de la región no reside dentro de la región , o si la
	% excentricidad excede el umbral $\varepsilon+{max}$ y $\frac{l_{max}}{s}$ excede el umbral $r_{max}$ ~\brackcite{koch2011pothole}.

% \begin{center}
% 	$type(R)$ := $\left\{\substack{\text{Es candidato a representar la sombra de un bache} :\;\;P_{cent}\in{R}\;\; \vee  (\varepsilon > \varepsilon_{max} \wedge (\frac{l_{max}}{s} > r_{max})) \\
% 	\noindent
% 	\text{Es candidato a representar un bache completo} : \text{en otro caso}}\right .$
% \end{center}


	% Para aproximar la forma elíptica de un bache basado en la forma detectada se procede a realizar los siguientes pasos ~\brackcite{koch2011pothole}:

	% \begin{enumerate}
	% 	\item Utilizan adelgazamiento morfológico para reducir la región de sombra  a su trazo o esqueleto mínimamente conectado.
	% 	\item Los puntos de ramificación del esqueleto se identifican y son conectados en orden para determinar el camino principal de la región de sombra.
	% 	\item Si el número de puntos finales del esqueleto es inferior a 5, significa que el límite de la forma es casi uniforme, por lo que el esqueleto
	% 		completo define el camino principal.
	% 	\item Finalmente, teniendo como base el camino principal se utiliza la regresión elíptica para aproximar una elipse.
	% \end{enumerate}

	% Lo anterior describe como  extraer la forma de la región interior de un bache. Si se determina que la forma de una región candidata a defecto es un
	% bache, entonces el valor aproximado de la elipse define la región interior $R_i$. En el caso de una supuesta región de bache completa, la región de
	% defecto completa se utiliza para definir $R_i$.

	\subsection{Extracción de la textura}


% Es una de las propuestas es segmentar la  imagen en regiones defectuosas y no defectuosas ~\brackcite{koch2011pothole}
% Luego utilizar un proceso de  \emph{thresholding} sobre un histograma de la forma de la imagen (histogram shape-based thresholding)

% Con base en las propiedades geométricas de una región defectuosa, la
% la forma potencial del bache se aproxima utilizando adelgazamiento morfológico y regresión elíptica. ~\brackcite{koch2011pothole}


% Posteriormente, se extrae la textura dentro de una posible forma defectuosa y se compara con la textura del pavimento circundante 
% sin defectos para determinar si la región de interés representa una forma real.

% bache.ndear el pavimento sin defectos para determinar si la región de interés representa un real bache

% En las imágenes de la superficie del pavimento, la información de color, en particular
% Los valores RGB no son imprescindibles a la hora de realizar la segmentación proceso con respecto a 
% la detección de defectos. Por lo tanto, el primer paso es transformar imágenes en color originales en 
% imágenes a escala de grises.

% Para cumplir con la primera característica identificada, las regiones más oscuras que indican defectos deben separarse de la parte posterior.
% suelo dentro de cada imagen de pavimento.

% Después de llenar los agujeros dentro de las regiones restantes del binario
% imagen B, regiones pequeñas (p. ej., grietas cortas, artefactos), regiones que

% tener una forma lineal (por ejemplo, grietas largas, juntas, cortinas de bordillo) y re-
% Se asumen giones, que están conectados al límite de la imagen.
% Para que no queden baches y así se eliminen.


% De acuerdo con la observación de los autores de que la región dentro de una olla
% el agujero es principalmente granular, los llamados filtros puntuales se utilizan para crear
% respuestas de filtro alto

\section{Detección utilizando series temporales}
	Los primeros trabajos que utilizaron las series temporales que se generan como resultado del muestreo de sensores como el \textbf
	{acelerómetro} y el \textbf{giroscopio}, para determinar la presencia de baches en la carretera, se basaron en el uso de técnicas
	de establecimiento de umbrales.\\

	\subsection{Enfoques utilizando métodos heurísticos}
		Entre estos primeros trabajos que propusieron varios métodos heurísticos basados en umbrales, todos tienen como premisa
		la idea de que condiciones anómalas de la carretera son reflejadas en características de los datos de aceleración \brackcite
		{mohan2008nericell, mednis2011real, eriksson2008pothole}.\\

		\emph{Mednis et al.} ~\brackcite{mednis2011real} propusieron los siguientes métodos:\\

		\begin{itemize}
			\item  \emph{\textbf {Z-Thresh}}: Establecer unos umbrales para los valores del acelerómetro en el eje Z en un instante de tiempo.\\
			\item \emph{\textbf {Z-DIFF}}: Establecer unos umbrales para la diferencia entre 2 valores consecutivos del acelerómetro.\\
			\item \emph{\textbf {STDEV(Z)}}: Establecer unos umbrales para la desviación estándar de los valorres del acelerómetro en
				el eje Z durante un intervalo de tiempo.\\ 
			\item \emph{\textbf {G-ZERO}}: Este método se basa en la idea de que cuando el vehículo interactúa con alguna anomalía, existe un
				instante de tiempo donde las lecturas del acelerómetro en los 3 ejes son cercanas a 0 y menores que cierto umbral prefijado. Lo 
				cual podría ser indicativo de que el vehículo está en caída libre en un pequeño intervalo de tiempo.
		\end{itemize}

		\emph{Mohan et al.} ~\brackcite{mohan2008nericell} usan una idea idéntica al \emph{\textbf{Z-Thresh}} y otra en la que buscan
		una caída sostenida de la aceleración en el eje Z por debajo de cierto umbral, durante cierta cantidad de tiempo(\emph{\textbf{Z-SUS}}).\\

		% Los 3 primeros métodos requieren información acerca de la orientación del eje Z imaginario del dispositivo encargado de recopilar la
		% información. Para capturar datos preliminares utilizaron un collar \emph{LynxNet} a una frecuencia de 100Hz. Para la evaluación del
		% sistema utilizaron 4 dispositivos móviles (Samsung i5700, Samung Galaxy S, HTC Desire, HTC HD2), con los que recolectaron más datos
		% y los procesaron con el objetivo de asignar los valores óptimos a cada uno de los umbrales característicos de cada uno de
		% los 4 métodos propuestos, así como el tamaño de la ventana en caso de \textbf{STDEV(Z)}.\\

		% anomalía (un bache en este caso) las ruedas descienden en el hueco ocasionando una caída sostenida en la aceleración en el eje Z,
		% hasta que alcanzan el fondo del bache y en ese momento se produce un pico bien elevado de aceleración en el eje Z. A altas velocidades
		% ese pico al final del evento es bien prominente, sin embargo cuando se va a muy poca velocidad no se nota prácticamente, pero la caída
		% sostenida en la aceleración en el eje Z al entrar al bache si se mantiene, por lo que propusieron 2 métodos para detectar una anomalía
		% dependiendo de la velocidad $v$ a la que se desplaze el vehículo:\\

		% \begin{itemize}
		% 	\item \textbf {v > 25 km/h}:  Utilizan un método idéntico al \textbf {Z-THRESH}, o sea, buscan un valor bien prominente de
		% 		aceleración en un instante de tiempo determinado, que sea mayor que cierto umbral prefijado.\\
		% 	\item \textbf {v < 25 km/h}:  Utilizan un método que llamaron \textbf {Z-SUS} que busca una caída sostenida en la aceleración
		% 		en el eje Z, por debajo de un cierto umbral prefijado, durante cierta cantidad de tiempo.\\
		% \end{itemize}


		El algoritmo para la detección de anomalías en la carretera propuesto por \emph{Eriksson et al.} ~\brackcite{eriksson2008pothole} sigue
		esta misma idea de establecer umbrales pero es un poco más elaborado, ya que usa una 5 filtros donde cada filtro está diseñado
		para rechazar uno o más de un evento que no sea bache:\\

		\begin{itemize}
			\item \emph{\textbf {Velocidad}}: Las ventanas donde el carro no se está moviendo o se está moviendo a muy poca velocidad son ignoradas. 
				De esta forma se rechazan eventos como un tirón en la puerta cuando se baja el pasajero del vehículo.\\
			\item \emph{\textbf {High-pass}}: Los filtros de \emph{high-pass} eliminan las componentes de baja frecuencia de la señal de aceleración
				en los ejes z y x, que pueden indicar giros, frenazos, etc.\\
			\item  \emph{\textbf {Z-peak}}: Los picos de aceleración en el eje Z es la principal característica de las anomalías significativsas de
				la carretera. Este filtro rechaza todas las ventanas con un pico (absoluto) en la aceleración menor cierto umbral $t_z$.\\
			\item \emph{\textbf {xz-ratio}}: La aceleración en el eje x puede ayudar a identificar anomalías que abarcan el ancho de la carretera
				y por lo tanto impactan ambos lados del carro de igual manera.  Asumiendo que el bache solo impacta un lado del carro, un verdadero
				evento de bache con un un gran pico de aceleración en el eje z debería producir un pico significante en el eje x dentro de alguna
				ventana.\\
			\item \emph{\textbf {speed vs ratio }}:	A altas velocidades, incluso pequeñas anomalías en la carretera pueden crear altos picos en las
				lecturas de aceleración. Este filtro rechaza ventanas donde el pico de aceleración en el eje Z es menor que un factor $t_s$ veces la
				velocidad a la que va el vehículo.
		\end{itemize}

		% se basa en que condiciones
		% anómalas de la carretera son reflejadas en características de los datos de aceleración. El problema para identificar baches a partir
		% de lecturas del acelerómetro es bastante complejo debido a la amplia variación en las condiciones de la carretera y el comportamiento
		% del conductor. La mayoría de las anomalías pueden ser caracterizadas como eventos de alta energía en la señal de la aceleración, por si
		% sola la energía de la señal no es suficiente como un criterio de detección debido a que muchos eventos de alta energía pueden no
		% considerarse anomalías en la carretera. Muchos accesorios de la carretera como los cruces de trenes, las juntas de expansión, los
		% eventos de alta energía que pueden ser causados por los pasajeros cuando le dan un fuerte tirón a la puerta del vehículo o si el conductor
		% frena de manera repentina.
		% Los datos recopilados son divididos en 256 ventanas pues los eventos que interesan son generalmente eventos de
		% corta duración. 

	\subsection{Enfoques utilizando aprendizaje de máquinas}
		\subsubsection{Algoritmos}
			Con respecto a los trabajos existentes que utilizan el aprendizaje de máquinas existen varios. Entre los modelos supervisados
			más utilizados están \emph{Support Vector Machines}(\textbf{SVM}) ~\brackcite{tai2010automatic, carlos2018evaluation, seraj2015roads,
			el2018towards, cong2013applying, perttunen2011distributed, gonzalez2017learning}, \emph{Decision Trees} ~\brackcite{el2018towards,
			gonzalez2017learning}, \emph{Random Forest} ~\brackcite{el2018towards, zheng2020fused}, \emph{K-Nearest Neighboors}(\textbf{KNN})
			~\brackcite{el2018towards, zheng2020fused, gonzalez2017learning} y \emph{Artificial Neural Networks}(\textbf{ANN}) \brackcite
			{kulkarni2014pothole, pawar2020efficient, gonzalez2017learning}. \emph{Zheng et al.} \brackcite{zheng2020fused} agregan a este proceso
			\textbf{DTW}(\emph{Dynamic Time Warping}) que se utiliza para determinar similitud entre 2 series temporales que puedan ser obtenidas
			a distintas velocidades. 

		\subsubsection{Selección y extracción de características}
			Entre los principales características utilizadas en estos modelos de aprendizaje de máquina se encuentra, como era de esperar, el valor 
			del \textbf{acelerómetro} en los 3 ejes, pues este dato es el más característico de una bache en la carretera. De este se derivan otras
			características estadísticas como la media, desviación estándar, varianza, coeficiente de variación, diferencia entre valor mínimo y
			máximo, mediana, moda y rango entre cuartiles \brackcite{carlos2018evaluation, seraj2015roads, el2018towards, perttunen2011distributed,
			pawar2020efficient}. Incluso características, como por ejemplo, cuantas veces cierta característica estadística sobrepasó cierto umbral
			establecido \brackcite{carlos2018evaluation}, una idea basada en los primeros trabajos en relación a este problema y que utilizaron
			heurísticas basadas en umbrales.\\
			\indent Algunos autores proponen el uso de la señal del \textbf{giroscopio} y la extracción de características estadísticas a partir de esta,
			pues se encarga de medir la velocidad de giro del dispositivo, y los baches y las anomalías en general, deben producir lecturas anormales
			de este sensor. De esta forma se podría tener más información que permita mejorar el rendimiento de los algoritmos de aprendizaje
			supervisado \brackcite{seraj2015roads, pawar2020efficient}.\\
			\indent Debido a que este enfoque se basa en el uso de señales digitales, varias propuestas han utilizado herramientas del campo de procesamiento
			de señales digitales para mejorar y enriquecer la calidad de sus datos y sus modelos. Características extraídas utilizando la Transformada
			Rápida de Fourier(\textbf{FFT}) como frecuencia media, mediana de la frecuencia y energía de las bandas de frecuencia \brackcite{seraj2015roads,
			perttunen2011distributed} y otras que fueron extraídas utilizando \emph{Wavelet Decomposition} específicamente \emph{Stationary Wavelet
			Decomposition} \brackcite{seraj2015roads} o \emph{Wavelet Packet Decomposition} \brackcite{cong2013applying}.\\
			\indent En algunos de los artículos se propone analizar la relevancia de los características seleccionadas a la hora de discriminar entre baches y otros
			segmentos de carretera, utilizando algún proceso de selección de características como \emph{Forward Selection}, \emph{Backward Selection}
			o \emph{Principal Components Analysis}(\textbf{PCA}) \brackcite {cong2013applying}.\\
			\indent \emph{Gonzales et al.} \brackcite{gonzalez2017learning} usan las características para construir una representación por intervalos de la serie
			temporal empleando la técnica de \emph{Bag of Words}, de la cual extraen varias características nuevas (palabras) que describen cada fragmento
			en el que segmentaron la serie temporal. Algunos de los autores plantearon que la velocidad puede influir de forma directa en como se reflejan
			los valores del acelerómetro en el eje Z en los datos, y dicho valor puede cambiar drásticamente si se interactúa con una anomalía a velocidades
			distintas \brackcite{eriksson2008pothole, perttunen2011distributed, mednis2011real}. \emph{Perttunen et al.} \brackcite{perttunen2011distributed}
			propusieron un método para eliminar la dependencia de la velocidad de dicha característica. 

		\subsubsection{Etiquetado de los datos}
			Para el uso de estos modelos supervisados es necesario tener datos etiquetados. Debido a que las anomalías y los baches ocurren por una 
			fracción de segundo determinar exactamente el momento en el que ocurre es un proceso complicado. Algunos autores propusieron etiquetar
			los datos utilizando vídeo y audio capturado durante el recorrido identificando cada evento que ocurriese durante el viaje \brackcite
			{seraj2015roads}. Otros utilizaron solamente vídeos capturados durante el proceso de recolección de datos para etiquetarlos \brackcite
			{el2018towards}.\\
			\indent \emph{Gonzales et al.} \brackcite{gonzalez2017learning} etiquetaron las anomalías de la siguiente forma, cuando se van acercando a la
			anomalía ponen el dispositivo a recolectar los datos y una vez pasan la anomalía detienen la recolección de datos y asignan una etiqueta
			a dicha serie temporal. \emph{Tai et al.} \brackcite{tai2010automatic} etiquetaron sus datos realizando grabaciones de audio durante el
			viaje recolectando la voz de la persona encargada de notificar la presencia de una anomalía.


	\subsection{Purificación y segmentación de la señal}
		Otros aspecto a tener en cuenta a la hora de intentar solucionar este problema es que las señales digitales están sujetas a ruido que puede ser
		ocasionado por varias razones, y que dificulta la tarea en cuestión.\\
		\indent Muchos autores proponen utilizar métodos de procesamiento de señales digitales para purificar la señal y mejorar su calidad. Métodos como \emph
		{Fast Fourier Transform}(\textbf{FFT}), \emph{Wavelet Decomposition} \brackcite {el2018towards}, \emph{Wavelet Discrete Transform} \brackcite
		{seraj2015roads} y \emph{high-pass filters} y \emph{low-pass filters} p\brackcite{eriksson2008pothole, kulkarni2014pothole} para eliminar
		componentes de baja y alta frecuencia respectivamente.\\
		\indent También es importante determinar de que forma segmentar la señal para su análisis. Hay propuestas que segmentan la señal en 256 ventanas
		\brackcite{eriksson2008pothole}, en ventanas con duración de 1 segundo \brackcite{el2018towards}, ventanas deslizantes con duración de 2.5,
		segundos con el 66\% de las ventanas solapadas \brackcite{seraj2015roads} y ventanas deslizantes con varias duraciones entre 0.5 y 2 segundos
		\brackcite{perttunen2011distributed}.\\
		\indent Una interesante propuesta por \emph{Zhenf et al.} \brackcite{zheng2020fused} sugiere que la mayoría de los artículos con respecto al tema no
		toman en consideración el hecho que la gran mayoría de la carreteras que existe en el mundo no posee anomalías, y que la aplicación de técnicas
		de aprendizaje de máquinas utizando una ventana deslizante a ciegas puede disminuir considerablemente la precisión y la rapidez del proceso de
		entrenamiento, por lo que proponen un método para determinar de forma dinámica el tamaño de la ventana a considerar.\\
		
	\subsection{Ubicación del dispositivo en el vehículo}
		Debido a que los dispositivos utilizados en estos experimentos tienen los 3 ejes prefijados, es necesario colocarlos de tal manera que los
		datos recopilados tengan sentido, o sea que se conozca la orientación del eje Z del dispositivo o tener alguna forma de estimar dicha orientación.
		Algunos autores colocaron el dispositivo en el parabrisas \brackcite{pawar2020efficient, seraj2015roads}, en el asiento delantero, en la parte
		central del asiento trasero del vehículo y en el maletero \brackcite{el2018towards}
		
	\subsection{Frecuencia de muestreo}
		La frecuencia de muestreo es un factor importante, si es muy alta se introduciría ruido innecesario en los datos, y si es muy baja y no se podrá
		obtener toda la información necesaria para resolver el problema en cuestión. Algunos autores utilizaron frecuencias de muestreo de 47 y 93Hz \brackcite
		{seraj2015roads}, 50Hz \brackcite{carlos2018evaluation}, 25Hz \brackcite{tai2010automatic}, 310Hz \brackcite{mohan2008nericell}, 380Hz para el 
		\textbf{acelerómetro} y 1Hz para el \textbf{GPS} \brackcite{eriksson2008pothole}, 26, 47, 52, 98 y 100Hz \brackcite{mednis2011real}, 100Hz para el 
		\textbf{acelerómetro}, 200Hz para el \textbf{giroscopio} y 1Hz para el \textbf{GPS} \brackcite{el2018towards}.

	\subsection{Geolocalización del bache utilizando GPS}
		Pocos de los autores se plantean resolver el problema de identificar la ubicación del bache utilizando el \textbf{GPS}. Debido a que el
		error en la medición del \textbf{GPS}(aproximadamente de 5m a la redonda como mediana) es significativamente mayor que el tamaño de un bache
		común, la ausencia de una detección en una localización particular no siempre es un indicativo de que no haya anomalías en esta. No es
		posible determinar puramente por la localización del \textbf{GPS} en el momento en que las ruedas del vehículo hacen contacto con alguna
		anomalía de la carretera.\\
		\indent Los conductores usuales intentan evitar los baches, por lo que la probabilidad de detectar una anomalía en la carretera es menor que la
		que se debe de esperar de una distribución sin sesgo del área que se intenta mapear \brackcite {eriksson2008pothole}. Dado que la frecuencia
		de captura de la señal del \textbf{acelerómetro} se hace mucho más seguido que la del \textbf{GPS}, se propone estimar la localización del
		vehículo cuando ocurra la lectura $l_i$ del \textbf{acelerómetro} usando interpolación lineal entre lecturas de \textbf{GPS} \brackcite
		{eriksson2008pothole}.

		% \brackcite{carlos2018evaluation} de realizar una comparación de los métodos existentes en su momento, crearon un método en el que utilizaron
		% un enfoque supervisado con \emph{Support Vector Machines} (\textbf{SVM}) para entrenar un modelo que decidiera utilizando varios \textbf
		% {características} (basadas principalmente en la aceleración en el eje Z), sí existía o no una anomalía. características como la media,
		% desviación estándar, varianza, coeficiente de variación y diferencia entre valor mínimo y máximo. otras 4 características utilizados
		% para otorgar un valor de confianza a 4 de los características previamente mencionados excepto a la varianza, comparándolos con un cierto
		% valor umbral. Esta idea fue tomada teniendo en cuenta una serie de artículos con respecto al tema(~\brackcite{mednis2011real}), con el
		% objetivo de mejorar el rendimiento del clasificador. Otro de los características es la cantidad de veces que un feature estadístico
		% sobrepasó el valor \emph {threshold} concebido. Completan los 12 características utilizados la suma de los valores de confianza, así
		% como el valor de confianza correspondiente. Cabe desctacar que implementaron una plataforma llamada Pothole Lab con el objetivo de crear un
		% sitio web público donde tener acceso a data sets robustos y curados. Los datos que utilizaron fueron obtenidos de este mismo sitio y fueron
		% capturados utilizando un Moto G Android a una frecuencia de 50Hz.\\

		% ~\brackcite{seraj2015roads} en su propuesta, incorporaron además los datos del giroscopio con el objetivo de obtener más características
		% que permitieran mejorar la calidad del clasificador. Además, también realizaron un proceso para eliminar el ruido y mejorar la calidad de
		% las señales obtenidas con los sensores de los dispositivos utilizados para capturar los datos. Separaron las señales en ventanas de 2.5
		% segundos y extrajeron de ahí varios características para el proceso de entrenamiento. características de dominio temporal,
		% características de dominio de frecuencia y características extraídos de la transformada de wavelet son los que conforman los
		% vectores. Finalmente entrenaron un modelo utilizando una \textbf{SVM} con los características que extra jeron de los datos, razón por
		% la cual etiquetaron sus datos utilizando vídeos y audios grabados durante la recolección de los datos. El dispositivo que utilizaron para la
		% recolección de los datos fue un Samsung Galaxy S2 y recopilaron datos a frecuencias de 47Hz y 93Hz.\\

		% ~\brackcite{el2018towards} como en el artículo previamente mencionado, llevaron a cabo un extenso proceso, separando las señales de los
		% sensores por ventanas de 1 segundo y generando vectores de características bien extensos en cada una de estas ventanas. Además llevaron
		% a cabo un proceso de eliminación de ruido de las señales probando varios métodos conocidos en el campo del procesamiento de señales
		% digitales como la transformada de Fourier o la transformada de Fourier por ventanas discretas (\textbf{WDFT} por sus siglas en inglés),
		% pero finalmente utilizaron transformadas de wavelets para purificar la señales y obtener datos más confiables con los que llevar a cabo
		% el proceso de entrenamiento del modelo de aprendizaje de máquinas. Se decantaron por un enfoque supervisado, por lo que etiquetaron de
		% antemano los datos utilizando vídeos capturados durante el mismo proceso de recolección de datos, métodos como \emph{K-Nearest Neighboors}
		% (\textbf{KNN}), \emph{\textbf{Decision Trees}}, \textbf{SVM} y \emph{ensembles} de clasificadores fueron probados, obteniendo los mejores
		% resultados con las \textbf{SVM} y los \emph{bagged decision trees}. Utilizaron características estadísticos, de dominio temporal, de dominio
		% de frecuencia y de dominio frecuencia-temporal. Entre los características que utilizaron están media, mediana, moda, desviación estándar,
		% varianza, rango entre cuartiles, media de la frecuencia, la mediana de la frecuencia, etc.\\

		% //\\// Pendiente //\\//
		% - Añadir detalles sobre el uso del GPS.
		% - Añadir detalles sobre si es multi-class classification o binary-classification
		% - Añadir detalles sobre el uso de distintos vehículos.
		% - Añadir detalles sobre la ubicación de los dispositivos en el vehiculo.


		
		% \brackcite{eriksson2008pothole}	Dado que la frecuencia de captura de la señal del acelerómetro se hace mucho más seguido que la del GPS,
		% proponen estimar la localización del vehículo cuando ocurra la lectura $l_i$ del acelerómetro usando interpolación lineal entre lecturas
		% de GPS. A pesar que ciertos tipos de anomalías pueden producir alta energía, no siempre que se produzca esto representa una carretera en
		% mal estado. Los cruces de ferrocarriles, reductores de velocidad y otros equipamientos bien conocidos embebidos en la carretera pueden
		% producir una alta energía y no se consideran anomalías de la carretera. Para esto proponen tener una lista negra con los localizaciones
		% de las anomalías que caigan en alguno de estos equipamientos embebidos en la carretera cuya información se puede obtener de y luego
		% remover los mismos. Debido a que el error en la medición del GPS (alrededor de 5m a la redonda como media) es significativamente mayor
		% que el tamaño de un bache común, la ausencia de una detección en una localización particular no siempre es un indicativo de que no haya
		% anomalías en esta []. No es posible determinar puramente por la localización del GPS en el momento en que las ruedas del vehículo hacen
		% contacto con alguna anomalía de la carretera. Los conductores usuales intentan evitar los baches, por que la probabilidad de detectar una
		% anomalía en la carretera es menor que la que se debe de esperar de una distribución sin sesgo del área que se intenta mapear.\\

		% ~\brackcite{cong2013applying} para la extracción de características descomponen la señal usando \textbf{WPD}(\emph{Wavelet Packet Decomposition})
		% usando sucesivamente filtros de \emph{low-pass} y de \emph{high-pass}. Donde \textbf{WPD} es llevado a cabo aplicando de forma iterativa filtros
		% espejos en cuadratura y seguido por submuestreo. La selección de características basadas en aprendizaje de máquinas puede ser asistida para
		% reducir la demanda computacional para la clasificiación. La selección de características está diseñada para encontrar las características que
		% hacen una mejor discriminación de los baches y los segmentos normales en estudio. Para la selección de características se probaron con cuatro
		% métodos: \textbf{BS}(\emph{Backward Selection}), \textbf{FS} (\emph{Forward Selection}),  y \textbf{PCA} usando un número diferente de \emph
		% {features} seleccionadas. \textbf{PCA} fue el que arrojó mejores resultados cuando el número de características es mayor que 5, mientras \textbf
		% {FS}(\emph{Forward Selection}) es mejor cuando el número de características es mayor que 2 y menor que 6.\\
		% Para la clasificicación usaron \textbf{SVM} one-class classification con kernel RBF con parámetros  $\nu = 0.01$  $\gamma = 0.00002$. \textbf
		% {SVM} fué entrenado con el 70 \%  de los datos (1234 segmentos) y el conjunto de entrenamiento fue escogido de manera aleatoria. El resto de
		% los datos  (530 segmentos) y en todos los 21 segmentos de anomalías fueron usados para los tests de precisión del modelo \textbf{SVM} construido.\\

		% ~\brackcite{kulkarni2014pothole} proponen un sistema que detecta los baches, registra su ubicación, un crea un documento que puede utilizarse
		% para cargarlo en un servidor centralizado o enviarlo  a las autoridades competentes inmediatamente.  Cuando el usuario inicia su viaje, lanza
		% la aplicación Android de detección de baches. La aplicación, que tiene como complemento del algoritmo en funcionamiento, detecta los baches en
		% las carreteras mientras el usuario está conduciendo. Supervisa los cambios en la aceleración. La aplicación añáde la hora actual, las coordenadas
		% geográficas y las estadísticas de baches al registro de eventos. Cuando el usuario  finalice el recorrido pulsa "Stop"  y se le presenta el
		% registro de eventos. Este registro debe mantenerse en la base de datos.  El algoritmo que proponen es el siguiente:\\\\
		% \noindent

		% \begin{itemize}
		% 	\item  Un filtro de \emph{high-pass} para remover las componentes de baja frecuencia de la señal de aceleración en los eje x y z.  El
		% 		filtro de \emph{high-pass} elimina el desplazamiento de la gravedad. El alpha usado es de 0.8 \\\\
		% 	\begin{align*}
		% 		\alpha = 0.8 \\
		% 		gravity_{x} = \alpha * gravity_x + (1-\alpha) *event.values_{x} \notag\\
		% 		gravity_{y} = \alpha * gravity_y + (1-\alpha) *event.values_{y} \notag\\
		% 		gravity_{z} = \alpha * gravity_z + (1-\alpha) *event.values_{z} \notag\\	
		% 	\end{align*}
					
		% 	Luego el effecto de \emph{high-pass} para la eliminación de las componentes de baja frecuencia
			
		% 	\begin{align*}
		% 		acceleration_{x} =  event.values_{x} - gravity_{x}\\
		% 		acceleration_{y} = event.values_{y} - gravity_{y}\\
		% 		acceleration_{z} = event.values_{z} - gravity_{z}
		% 	\end{align*}

		% 	\item Los picos de aceleración en el eje Z es una de las características principales de las anomalías en la carretera. Este filtro rechaza
		% 		todas las ventanas con un pico de aceleración en la componente z menor menor que un umbral \textbf{tz} ( o sea rechaza la
		% 		lectura si $accel_z < tz$ ). \item Un verdadero evento de bache con una larga aceleración en la componente z debe producir un pico
		% 		significativo en el eje x. Este filtro rechaza todas las ventanas con un pico de aceleración en la componente z menor que  el producto
		% 		del umbral en el eje x \textbf{tx} por  el pico de aceleración (o sea si tx es el umbral en el eje x rechazaría las
		% 		lecturas que $accel_x < tx * accel_z$). \item A altas velocidades pequeñas anomalías pueden crear altos picos en las lecturas de
		% 		aceleración. Este filtro rechaza las ventans donde los picos de aceleración en el eje z son menores que un factor ts veces la velocidad
		% 		a la que se viaja(O sea asumiendo que el factor de velocidad es ts, y \emph{speed} la velocidad actual a la que viaje el vehículo rechaza
		% 		las lecturas que cumplan $accel_z < ts * speed$). \item  Si todas las condiciones anteriores se cumplen entonces se considera un bache o
		% 		en caso contrario no. \item  Se usa una red neural para mejorar la eficiencia y precisión de la detección de baches. Los parámetros de
		% 		la red neuronal usados son los siguientes:\\

		% 	\begin{enumerate}
		% 		\item Número de capas de entrada: 3
		% 		\item Input 1 : aceleración en el eje x
		% 		\item Input 2 : aceleración en el eje x
		% 		\item Input 3 : aceleración en el eje x
		% 		\item Número de neuronas ocultas: 6
		% 		\item Número de capas de salidas: 1
		% 		\item Output 1 : Decidir si es un bache
		% 		\item Función de activación: sigmoidal
		% 		\item Algoritmo usado: \emph{backpropagation}
		% 	\end{enumerate}
			
		% \end{itemize}

		% ~\brackcite{zheng2020fused} se refieren a que la mayoría de los artículos con respecto al tema no toman en consideración el hecho que la gran
		% mayoría de la carreteras que existe en el mundo no posee anomalías, y que la aplicación de técnicas de aprendizaje de máquinas utizando una
		% ventana deslizante a ciegas puede disminuir considerablemente la precisión y la rapidez del proceso de entrenamiento. Para esto plantean que
		% una anomalía comienza con una señal normal y luego ocurre un pico de aceleración en el eje Z que excede un umbral superior y luego
		% cae y excede un umbral inferior o viceversa, y finalmente la señal vuelve a estabilizarse entre esos dos umbrals. Ese
		% intervalo es el que consideran como el intervalo candidato en el que puede haber una anomalía, y para encontrarlo utilizan primero el método
		% heurístico de establecer un umbral superior e inferior en la aceleración y buscan el primer instante donde la aceleración sobrepasa
		% alguno de los umbrales y el último instante, y utilizan un método para encontrar los instantes de tiempo justo antes de que ocurriera
		% el primer evento y el instante de tiempo justo después de que ocurriera el último evento, de esta forma logran construir una ventana de tamaño
		% dinámico que contiene toda la información acerca de la anomalía, obteniendo como resultado un conjunto de ventanas con posibles anomalías en
		% la serie temporal. Luego crean un modelo con un \emph{Random forest} utilizando algunos características estadísticos de cada una de las ventanas
		% para filtrar ventanas que no constituyen anomalías reales, también proponen tratar al vehículo como un modelo de vibración de un nivel de
		% libertad para lo cual diseñan varios experimentos. Como resultado de esto llegaron a la conclusión de que la varianza de la aceleración en el
		% eje Z cuando el vehículo interactúa con una anomalía tendrá una relación casi lineal con la profundidad o altura de la anomalía. Utilizando
		% \textbf{KNN} y \textbf{DTW} (\emph{Dynamic Time Warping} que se utiliza para determinar similitud entre 2 series temporales que puedan ser
		% obtenidas a distintas velocidades), es que llevan a cabo el proceso de identificar los tipos de anomalías. Los autores comparan su propuesta
		% con otras hechas en los últimos 3 años obteniendo un mejor \emph{F1 score} al identificar los 3 tipos de anomalías que consideraron en cada
		% uno de los 3 \emph{datasets} que probaron.\\

		% ~\brackcite{perttunen2011distributed} explican que los sensores del teléfono móvil son reflejados en las señales de dos maneras: Primero, la
		% señal del \textbf{GPS} en el teléfono usado tenía muchco ruido. Segundo, las mediciones del \textbf{GPS} y de la aceleración fueron contaminadas
		% por ráfagas, que son mediciones registradas con la misma marca de tiempo. A continuación, se realizó el rechazo de valores atípicos de GPS y se
		% aplicó el filtro de Kalman a la latitud y la longitud para reducir aún más el ruido. Después la velocidad será estimamda por cada par consecutivo
		% de latitud y longitud. La señal resultante de velocidad fué sobremuestrada y filtrada usando límites físicos razonables para la aceleración del
		% vehiculo. Cuando las estimaciones de velocidad parecían lo suficientemente suaves, la señal estaba a menudo contaminada por una gran latencia en
		% comparación con la estimación de velocidad original (ruidosa) y la señal no llegaba a cero en las paradas de los vehículos. Para aliviar este
		% problema se usan dos correcciones. En primer lugar, se elimina una latencia determinada visualmente de la estimación de velocidad filtrada. En
		% segundo lugar, se aplica una fusión muy sencilla de la aceleración y la señal de \textbf{GPS}: la varianza de la norma de la aceleración se
		% calculó para la señal de aceleración y se examinó visualmente. Mediante un simple umbral fueron capaces de detectar un segmento de
		% la señal, donde la velocidad del vehículo era cero, o muy cercana a cero. A continuación, establecieron los segmentos correspondientes de la
		% señal de velocidad en cero y suavizaron la señal utilizando un filtro de Kalman. Los datos fueron seperados usando una ventana deslizante.
		% Experimentaron con pedazos desde 0.5 segundos hasta 2 segundos, esta escala se consideró aceptable para la tarea de reconocimiento de anomalías,
		% ya que la duración media de las anomalías era alrededor de 2 segundos. Por cada ventana, se determinaba el por ciento de la ventana cubierta
		% por anomalías (uno o más juntos). La extracción de características se realiza utilizando ventanas deslizantes de 2 segundos de longitud, con un
		% deslizamiento de 0.5 segundos. Varias características fueron extraídas de la señal de aceleración: la desviación estándar, la media, la varianza,
		% \emph{peok-to-peak}, \emph{signal magnitude area}, \emph{3-order autoregressive coeficcients}, \emph{til angles}, la raíz de la desviación
		% cuadrática media. Los valores absolutos de la correlación de señales entre todas las dimensiones son usados también, ya que se observó
		% visualmente muchas veces todas las señales de aceleración mostraban similar formas de onda en los segmentos de la anomalía. Se utilizaron
		% características basadas en \emph{Fast Fourier Transformation(FFT)} para incorporar información de frecuencias específicas. Esto se basó en
		% la suposición de que los baches producirían componentes de menor frecuencia en comparación a la vibración que se origina en el motor y en
		% la superficie normal de la carretera. La energía \textbf{FFT} se extrajo de 17 bandas de frecuencia para cada dirección de la aceleración
		% y los coeficientes cepstrales de frecuencia de mel en 4 bandas. Se utilizó el algoritmo de selección de características \emph{Backward
		% Selection} para seleccionar el conjunto de características óptimo tanto para los que eliminaron la dependecia de velocidad como para los
		% que no. Usaron un método para remover la dependecia de la velocidad en los características. Para clasificar las ventanas, que representan
		% pequeños segmentos de la carratera, se utilizó SVM con kernel RBF. El clasificador con el mejor g-means medio luego de varias corridas fue
		% seleccionado. Evaluaron 49 combinaciones de parámetros con \textbf{SVM} usando función de kernal radial (RBF), correspondientes a una
		% rejilla de valores de los parámetros $\gamma$ y C. Presentaron además un \emph{framework} de visualización para los resultados, parar
		% habilitar la inspección visual de, por ejemplo, los ejemplos cercanos al borde de las clases.\\

		% ~\brackcite{pawar2020efficient} utilizan un dispositivo móvil ubicado en el parabrisas del vehículo para capturar las señales del acelerómetro y el
		% giroscopio. Con esto construyen un vector de 24 características entre los cuales están los valores máximo, mínimo, la desviación estándar
		% y la media de los 3 ejes del acelerómetro así como los del giroscopio, luego estandarizaron los datos para disminuir el efecto de las anomalías
		% en los características y permite percibir mejor en un gráfico los picos en las lecturas causados por los baches. Finalmente proponen como
		% modelo una red neuronal con 2 capas ocultas y 1 capa de salida, cada una de las capas ocultas compuesta de una capa densa con función de activación
		% ReLU, permitiendo
		% \emph{backpropagation} de forma más rápida. Adicionalmente cuenta con una capa de \emph{dropout} con el objetivo de prevenir el \emph{overfitting},
		% asignando 0 como valor de forma aleatoria a un porciento de las entradas luego de cada actualización. La capa de salida tiene como función de activación
		% la sigmoidal. La función de pérdida es \emph{binary cross entropy} y la red neuronal es entrenada durante 140 épocas con un \emph{batch size} de 3.
		% Compararon su propuesta con otras similares pero modificando los hiperparámetros e incorporando un modelo conocido como \textbf{SMOTE}(\emph{Synthetic
		% Minority OverSampling Technique}) con el objetivo de contrarrestar el gran desbalance entre clases que hay en este problema en particular(bache y no
		% bache en este caso).\\

		% ~\brackcite{gonzalez2017learning} realizaron un riguroso proceso de experimentación con respecto al tipo y total de vehículos con
		% un total de 12, entre ellos camiones incluidos, además probaron con distintas posiciones del móvil dentro del vehículo. Para la recolección de datos 2
		% personas iban en el vehículo, uno de ellos encargado de etiquetar las anomalías de la siguiente forma, cuando se van acercando a la anomalía ponen el
		% dispositivo a recolectar los datos y una vez pasan la anomalía detienen la recolección de datos y asignan una etiqueta a dicha serie temporal. Su
		% propuesta consiste en una forma distinta de procesar los datos capturados con el acelerómetro del dispositivo móvil, que es utilizando lo que se conoce
		% como \emph{Bag of Words representation}, donde los datos son representados utilizando un histograma que refleja la cantidad de veces que se repite una
		% palabra en un documento determinado. Esta representación se ha utilizado en áreas como visión por computadoras, procesamiento del habla y procesamiento
		% de series temporales como es el caso de este problema en particular. El objetivo es crear un diccionario de palabras clave que describa la serie temporal
		% por fragmentos (un tamaño prefijado) y luego representar en un histograma la frecuencia con que aparece cada palabra clave en una serie temporal dada.
		% Para crear esta lista de palabras clave separan en una cantidad de tramos prefijada las series temporales de los datos recolectados y etiquetados, con
		% datos de estos tramos como las lecturas, la media, la varianza, y los valores máximo y mínimo corren \emph{k-means} con todos los datos de una clase para
		% cada una de las clases, con el objetivo de agrupar los tramos similares en \emph{clusters} y obtener $k$ centroides, cada uno de estos lo consideran una
		% palabra clave perteneciente a una de las clases en cuestión. De esta forma cada clase tiene $k$ centroides que son sus palabras clave y así se obtiene 
		% un diccionario de clase:palabras clave. Luego para construir el vector de características segmentan la señal por tramos y comparan cada tramo con cada
		% uno de las palabras clave en el diccionario utilizando distancia euclideana y asignando a cada tramo la palabra clave con menor distancia, de esta forma
		% cada tramo se convierte en una palabra clave y es convertido a un histograma donde cada característica constituye la cantidad de veces que se repite cada
		% palabra clave en esa serie temporal, y es este vector el que entregan al clasificador. Probaron con una gran variedad de clasificadores redes neuronales
		% artificiales(\textbf{ANN}), \textbf{KNN}, \emph{Naive Bayes}(\textbf{NB}), \emph{Random Forest} (\textbf{RF}), \emph{Decision Trees}(\textbf{DT}),
		% \textbf{SVM}, y \emph{Kernel Ridge}(\textbf{KR}).\\
