\\\\
\textbf{Forward Selection}: FS es uno de los métodos de selección de características básicos. La idea como en otros métodos de selección de características, 
es seleccionar un subconjunto de características, que produce resultados lo suficientemente precisos en comparación con los resultados con todas las 
características. Forward selection comienza con 0 características en el modelo. La primera característica es seleccionada probando cada característica
que seleccionando la característica que devuelva por ejemplo la mejor clasificación o el mejor f-value  en tests estadísticos (ANOVA por ejemplo).   
\\\\
\textbf{Backward Selection}: Otro método de selección de características es \textbf{backward selection}, el cual es el proceso opuesto comparado con 
el \textbf{forward selection}. Mientras el forward selection comienza con 0 características, backward selection comienza con todas las características.
En cada paso el algoritmo testea todas las características que quedan y remueve la característica que menos aporta mal.
\\\\
\textbf{Genetic Algorithms}: Los algoritmos genéticos son un grupo de modelos computacionales para buscar soluciones potenciales a un problema en específico usando
una estructura de datos basadada en cromosomas, lo cual está inspirado en la evolución. Un cromosoma es una serie de instrucciones que un algoritmo puede usar para
construir un nuevo modelo o función, como un problema de optimización o seleccionar un subsconjunto de características para SVM. Todas las características serán 
representadas como un vector binario de tamaño m donde m es el número de características. \textbf{'1'} significa que la característica es parte del subconjunto y 
\textbf{'0'} que no lo es.  El algoritmo puede ser considerado como un proceso de dos etapas. Comienza con una población actual donde los mejores cromosomas son 
seleccionados para crear una población intermedia. Recombinación y mutación son aplicados para crear la nueva población. El proceso de dos etapas continúa una generación
en la ejecución del algoritmo genético.  El algoritmo comienza con una población inicial de cromosomas. Típicamente la población inicial se escoge de manera aleatoria 
del conjunto original de datos. Luego cada cromosoma es evaluado y se le asigna un fitness value. Los cromosomas que representen una mejor solución para el problema 
objetivo dan un mejor fitness value que aquellos cromosomas que dan una solución peor. Mejores fitness value significa mejores oportunidades de reproducir. EL proceso 
de reproducción puede ocurrir a través de crossover, mutation, o operaciones de reproducción. 
\\\\
\textbf{PCA}: PCA es otro método usado usado frecuentemente para la extracción de features. El PCA busca las componentes principales no correlacionadas que describen 
las dependencias entre múltiples variables. Las componente principales se ordenan de tal forma que la primera componente explique la mayor cantidad de varianza en los
datos, y la segunda componente para la segunda mayor cantidad de varianza y asi será.  Debe ser notado que forward selection, backward selection y genetic algorithm 
no afectan los datos, y solo son métodos para elejir la mejor combinación de características, pero PCA es para nuevas características. 
