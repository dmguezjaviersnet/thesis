\begin{recomendations}
	Uno de los problemas con los que se tuvo que lidiar fue con la falta de datos. Por razones ajenas a la voluntad de los autores,
	entre ellas el poco tiempo con el que se contó para realizar el trabajo de campo, no se pudo recopilar un juego de datos con un 
	tamaño aceptable para los estándares requeridos por los modelos de aprendizaje de máquinas. Por esta razón, entre las propuestas
	que se considera que ayudarían mucho al proceso de recopilación de datos, también está la idea de automatizar el proceso de
	etiquetado de datos. De esta forma, se podría lidiar con grandes cantidades de datos, algo que es crucial para los modelos de
	aprendizaje de máquinas.

	Otra sugerencia, es plantear un sistema que permita capturar las señales de los sensores sin necesidad de que
	esté fijo en una posición. Esta idea puede facilitar el uso de la aplicación para capturar los datos, incluso por personas ajenas
	al proyecto. De esta forma cualquier persona podría ayudar a mejorar la propuesta, simplemente con utilizar la aplicación, colocarse
	el móvil en el bolsillo y montarse en cualquier vehículo para capturar datos.

	También se propone explorar el uso de técnicas de procesamiento de señales digitales, que permitan mejorar la calidad de las señales
	de los sensores para obtener datos más curados, e incluso utilizar otras características que son posibles de extraer con algunos de
	estos métodos. Otra de las ideas es explorar otras maneras de procesar las series temporales de los sensores para pasárselas a los
	modelos de aprendizaje de máquinas, como por ejemplo, analizar la serie temporal por ventanas deslizantes y tratar de encontrar los
	patrones que caracterizan el estado de la carretera. Se podría indagar además, si existen otros sensores móviles que pudieran, de
	alguna forma, contribuir a mejorar la propuesta.

	Para futuros trabajos, se propone también graficar las series temporales de los sensores en la aplicación móvil en tiempo real.
	Esto facilitaría el entendimiento del comportamiento de las señales digitales durante la captura de datos. Finalmente, como parte
	de la meta de tener una aplicación que permita monitorear el estado de la carretera, se propone incorporar un \emph{backend} al
	proyecto donde los usuarios de la aplicación puedan subir los datos que hayan capturado a un servidor, y de esta forma utilizar
	dicha información centralizada y accesible para mejorar aún más la calidad de la propuesta.
\end{recomendations}
