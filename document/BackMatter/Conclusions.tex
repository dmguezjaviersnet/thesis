\begin{conclusions}
	En este trabajo se planteó como objetivo general, proponer un modelo de aprendizaje de máquinas para la detección de baches
	utilizando los sensores embebidos en los dispositivos móviles. Para esto, se implementaron 2 programas como parte de la propuesta.
	Una aplicación móvil desarrollada en \emph{Flutter} para poder obtener datos utilizando los sensores integrados de un dispositivo
	móvil. Y un programa hecho en \emph{Python}, que utilizando los datos capturados con la aplicación, permitiera analizar la
	factibilidad de un algoritmo de aprendizaje de máquinas para detectar de forma automática los baches en la carretera. Además,
	fue necesario analizar el estado del arte acerca del problema de monitorear el estado de la carretera utilizando sensores móviles.
	Esto tuvo el objetivo de determinar que sensores serían los más apropiados para resolver este problema, y encontrar ideas y
	acercamientos previos, que pudieran servir de guía para la propuesta de solución.\\

	Se pudo comprobar el funcionamiento correcto de la aplicación móvil en varias rutas realizadas por la ciudad de La Habana, y 
	por tanto se pudieron recopilar varios datos. Estos sirvieron luego para realizar un exhaustivo proceso, en el que los datos 
	fueron pasados a un \emph{pipeline} de 4 pasos, con el cual se determinó entre varios modelos, el que mejor se ajustaba a los
	datos recopilados. Teniendo en cuenta los resultados obtenidos en este trabajo, donde se logró alcanzar un \emph{F1 score} de
	68\% y un \emph{accuracy} de 70\%, se llegó a la conclusión, de que un modelo de aprendizaje de máquinas para resolver el
	problema en cuestión es factible.
\end{conclusions}
