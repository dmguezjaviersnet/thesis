\begin{opinion}
    Los estudiantes Javier E. Domínguez y David O. De Quesada Oliva desarrollaron satisfactoriamente el trabajo de diploma titulado 
    “Enfoque de Machine Learning para la Detección de Baches”. En este trabajo los estudiantes propusieron un sistema para 
    la detección automática de baches en la carretera.

    La propuesta que presentan se basa en las señales que generan los sensores de un dispositivo móvil (en particular, el acelerómetro, 
    el giroscopio y el GPS) en un vehículo en movimiento. Para ello, una vez obtenida esta señal se detectan, utilizando diferentes algoritmos, 
    las anomalías que pueden ser catalogados como baches. Luego, a partir de un corpus también confeccionado por los estudiantes, 
    utilizaron algoritmos de aprendizaje de máquina para su clasificación. Posteriormente, realizaron un conjunto de experimentos a 
    partir de un conjunto de señales capturadas por los estudiantes. Con este ejercicio mostraron la viabilidad de su propuesta.


    Para poder afrontar el trabajo, los estudiantes tuvieron que revisar literatura científica relacionada con la temática así como soluciones 
    existentes y bibliotecas de software que pueden ser apropiadas para su utilización. Todo ello con sentido crítico, determinando 
    las mejores aproximaciones y también las dificultades que presentan.

    Todo el trabajo fue realizado por los estudiantes con una elevada constancia, capacidad de trabajo y habilidades, tanto de gestión, como 
    de desarrollo y de investigación. 

    Por estas razones pedimos que le sea otorgada a los estudiantes David O. De Quesada Oliva y Javier E. Domínguez la máxima calificación y, 
    de esta manera,  puedan obtener el título de Licenciado en Ciencia de la Computación.
\end{opinion}
