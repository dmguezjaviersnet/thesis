\begin{resumen}
	Desde hace años, la cantidad de personas con dispositivos móviles ha crecido de manera vertiginosa. El desarrollo
	tecnológico ha permitido que dicha tecnología llegue a cada vez más personas a nivel mundial y facilite ciertas actividades 
	de la vida cotidiana del ser humano. Acá en Cuba, el incremento de usuarios de dispositivos móviles ha sido notable en los últimos
	años, más aún desde la aparición de los datos móviles. Por otro lado, el mal estado de las carreteras en La Habana, sigue siendo un 
	tema complejo, al que no se le ha encontrado solución definitiva. Es en este problema, donde el mencionado incremento de dispositivos
	móviles en el país puede representar un posible factor que contribuya de alguna forma a solucionarlo. Estos dispositivos poseen 
	sensores integrados que permiten recopilar información que puede ser útil para monitorear el estado de la carretera de forma sencilla y 
	barata. Muchos autores en la literatura han propuesto varios métodos para llevar a cabo esta tarea, desde técnicas basadas en umbrales 
	hasta los más recientes enfoques utilizando aprendizaje de máquinas. En este trabajo se propone explorar la factibilidad
	de una propuesta, utilizando modelos de aprendizaje de máquina, para detectar baches en la carretera. Estos modelos utilizan características 
	generadas a partir de los datos recopilados con los sensores integrados de los dispositivos móviles, específicamente el
	\textbf{acelerómetro}, \textbf{giroscopio} y \textbf{GPS}. De esta forma, se abrió una línea de investigación que puede ser enriquecida
	y mejorada con futuros trabajos, y que tiene como meta final lograr una plataforma que pueda facilitar el monitoreo del estado de las
	carreteras en el país.
\end{resumen}

\begin{abstract}
	In recent years, the number of people with mobile devices has grown at a dizzying rate. Technological development has allowed
	said technology to reach to more and more people worldwide and facilitate certain activities of the daily life of the human
	being. Here in Cuba, the increase in device users mobile phones has been notable in recent years, even more so since the
	appearance of mobile data. On the other hand, the poor condition of the roads in Havana continues to be a complex issue, to
	which no definitive solution has been found. It is in this problem, where the aforementioned increase in mobile devices in
	the country can represent a possible factor that contributes in some way to solving it. These devices have integrated
	sensors that allow the collection of information that can be useful to monitor the state of the road in a simple and cheap
	manner. Many authors in the literature have proposed several methods to carry out this task, from threshold-based techniques
	to the most recent approaches using machine learning. This paper proposes to explore the feasibility of a proposal, using
	machine learning models, to detect potholes in the road. These models use features generated from data collected with the
	built-in sensors of mobile devices, specifically the \textbf{accelerometer}, \textbf{gyroscope} and \textbf{GPS}. In this way,
	a line of research was opened that can be enriched and improved with future work, and which has as its final goal to achieve a
	platform that can ease the monitoring of the state of roads in the country.
\end{abstract}
